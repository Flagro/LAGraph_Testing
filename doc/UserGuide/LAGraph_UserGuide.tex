\documentclass[11pt]{extbook}

\usepackage{geometry}

\usepackage{fancyvrb}
\usepackage{color}
\usepackage{graphicx}
\usepackage{fullpage}
\usepackage{verbatim}
\usepackage{tikz}
\usepackage{listings}
\usepackage[yyyymmdd,hhmmss]{datetime}
\usepackage{rotating}
\usepackage{authblk}
\usepackage{amsfonts}
\usepackage{amsmath}
\usepackage{amssymb}
\usepackage{todonotes}
\usepackage{titlesec}
\usepackage[mathlines]{lineno}
\usepackage{tabularx}
\usepackage{enumitem}
\usepackage{bm}
\usepackage{etoolbox}
\usepackage{pdflscape}
\usepackage{threeparttable}
\usepackage{hyperref}

%TGM:  Added these packages to fix underscore rendering
\usepackage{lmodern} 
\usepackage[T1]{fontenc}

\setcounter{secnumdepth}{3}
\setcounter{tocdepth}{3}

\titleformat{\paragraph}
{\normalfont\normalsize\bfseries}{\theparagraph}{1em}{}
\titlespacing*{\paragraph}
{0pt}{3.25ex plus 1ex minus .2ex}{1.5ex plus .2ex}

\newtoggle{assign}
\toggletrue{assign}

\newcommand{\qg}{\u{g}}
\newcommand{\qG}{\u{G}}
\newcommand{\qc}{\c{c} }
\newcommand{\qC}{\c{C}}
\newcommand{\qs}{\c{s}}
\newcommand{\qS}{\c{S}}
\newcommand{\qu}{\"{u}}
\newcommand{\qU}{\"{U}}
\newcommand{\qo}{\"{o}}
\newcommand{\qO}{\"{O}}
\newcommand{\qI}{\.{I}}
\newcommand{\wa}{\^{a}}
\newcommand{\wA}{\^{A}}

\begin{document}

\linenumbers

\newcommand{\kron}{\mathbin{\text{\footnotesize \textcircled{\raisebox{-0.3pt}{\footnotesize $\otimes$}}}}}
\newcommand{\grbarray}[1]{\bm{#1}}
\newcommand{\scalar}[1]{{#1}}
\renewcommand{\vector}[1]{{\bf #1}}
\renewcommand{\matrix}[1]{{\bf #1}}
\renewcommand{\arg}[1]{{\sf #1}}
\newcommand{\zip}{{\mbox{zip}}}
\newcommand{\zap}{{\mbox{zap}}}
\newcommand{\ewiseadd}{{\mbox{\bf ewiseadd}}}
\newcommand{\ewisemult}{{\mbox{\bf ewisemult}}}
\newcommand{\mxm}{{\mbox{\bf mxm}}}
\newcommand{\vxm}{{\mbox{\bf vxm}}}
\newcommand{\mxv}{{\mbox{\bf mxv}}}
\newcommand{\gpit}[1]{{\sf #1}}
\newcommand{\ie}{{i.e.}}
\newcommand{\eg}{{e.g.}}
\newcommand{\nan}{{\sf NaN}}
\newcommand{\nil}{{\bf nil}}
\newcommand{\ifif}{{\bf if}}
\newcommand{\ifthen}{{\bf then}}
\newcommand{\ifelse}{{\bf else}}
\newcommand{\ifendif}{{\bf endif}}
\newcommand{\zero}{{\bf 0}}
\newcommand{\one}{{\bf 1}}
\newcommand{\true}{{\sf true}}
\newcommand{\false}{{\sf false}}
\newcommand{\syntax}{{C Syntax}}

\newcommand{\Dinn}{\mbox{$D_{in}$}}
\newcommand{\Din}[1]{\mbox{$D_{in_{#1}}$}}
\newcommand{\Dout}{\mbox{$D_{out}$}}

\newcommand{\bDinn}{\mbox{$\mathbf{D}_{in}$}}
\newcommand{\bDin}[1]{\mbox{$\mathbf{D}_{in_{#1}}$}}
\newcommand{\bDout}{\mbox{$\mathbf{D}_{out}$}}

%\newcommand{\aydin}[1]{{{\color{orange}[Aydin: #1]}}}
%\newcommand{\scott}[1]{{{\color{violet}[Scott: #1]}}}
\newcommand{\tim}[1]{{{\color{teal}[Tim: #1]}}}
%\newcommand{\jose}[1]{{{\color{red}[Jose: #1]}}}
%\newcommand{\ben}[1]{{{\color{blue}[Ben: #1]}}}

\newcommand{\scott}[1]{}
%\newcommand{\tim}[1]{}

%\aydin{testing}
%\scott{testing}
%\tim{testing}
%\jose{testing}
%\ben{testing}

\renewcommand{\comment}[1]{{}}
\newcommand{\glossBegin}{\begin{itemize}}
\newcommand{\glossItem}[1]{\item \emph{#1}: }
\newcommand{\glossEnd}{\end{itemize}}

\setlength{\parskip}{0.5\baselineskip}
\setlength{\parindent}{0ex}

%\usepackage{draftwatermark}
%\SetWatermarkText{DRAFT}
%\SetWatermarkScale{2}

\renewcommand{\thefootnote}{\fnsymbol{footnote}}
\setcounter{footnote}{1}

%-----------------------------------------------------------------------------

\title{
The LAGraph User Guide
{\large Version 1.0 DRAFT} \\
{\normalsize \tim{Remember to update acknowledgements and remove DRAFT}}
}

\author{Tim Davis, Tim Mattson, Scott McMillan, and \color{red}{others from the LAGraph group who commit major blocks of time to write this thing}}

\date{Generated on \today\ at \currenttime\ EDT}

\maketitle


\renewcommand{\thefootnote}{\arabic{footnote}}
\setcounter{footnote}{0}

\vfill

Copyright \copyright\ 2017-2022 Carnegie Mellon University, Texas A\&M University, 
Intel Corporation, and other organizations involved in writing this document. 

Any opinions, findings and conclusions or recommendations expressed in 
this material are those of the author(s) and do not necessarily reflect 
the views of the United States Department of Defense, the United States 
Department of Energy, Carnegie Mellon University, Texas A\&M University
Intel Corporation or other organizations involved with this document.  

NO WARRANTY. THIS MATERIAL IS FURNISHED ON AN AS-IS BASIS. THE COPYRIGHT 
OWNERS AND/OR AUTHORS MAKE NO WARRANTIES OF ANY KIND, EITHER EXPRESSED 
OR IMPLIED, AS TO ANY MATTER INCLUDING, BUT NOT LIMITED TO, WARRANTY OF 
FITNESS FOR PURPOSE OR MERCHANTABILITY, EXCLUSIVITY, OR RESULTS OBTAINED 
FROM USE OF THE MATERIAL. THE COPYRIGHT OWNERS AND/OR AUTHORS DO NOT MAKE 
ANY WARRANTY OF ANY KIND WITH RESPECT TO FREEDOM FROM PATENT, TRADE MARK, 
OR COPYRIGHT INFRINGEMENT.

\vspace{1.5cm}

\vspace{2cm}


\vspace{1.5cm}


%[Distribution Statement A] This material has been approved for public release and unlimited distribution.  
%Please see Copyright notice for non-US Government use and distribution.

Except as otherwise noted, this material is licensed under a Creative Commons Attribution 4.0 license (\href{http://creativecommons.org/licenses/by/4.0/legalcode}{http://creativecommons.org/licenses/by/4.0/legalcode}), 
and examples are licensed under the BSD License (\href{https://opensource.org/licenses/BSD-3-Clause}{https://opensource.org/licenses/BSD-3-Clause}).

%\begin{abstract}
%\end{abstract}

\vfill

\pagebreak
\tableofcontents
\vfill
\pagebreak

%-----------------------------------------------------------------------------

\phantomsection
\addcontentsline{toc}{section}{List of Tables}
\listoftables
\vfill
\newpage

\phantomsection
\addcontentsline{toc}{section}{List of Figures}
\listoffigures
\vfill
\newpage

%-----------------------------------------------------------------------------

\phantomsection
\addcontentsline{toc}{section}{Acknowledgments}
\section*{Acknowledgments}

This document represents the work of the people who have served on the LAGraph
Subcommittee of the GraphBLAS Forum.

Those who served as LAGraph API Subcommittee members are (in alphabetical order):
\begin{itemize}
\item David Bader (New Jersey Institute of Technology)
\item Tim Davis (Texas A\&M University)
\item Jim Kitchen (Anaconda)
\item Roi Lipman (redis Labs)
\item Timothy G. Mattson (Intel Corporation)
\item Scott McMillan (Software Engineering Institute at Carnegie Mellon University)
\item Michel Pelletier (Graphegon Inc)
\item Gabor Szarnyas (wherever)
\item Erick Welch (Anaconda)

\end{itemize}

The LAGraph Library is based upon work funded and supported in part by:
\begin{itemize}
\item Intel Corporation
\item Department of Defense under Contract No. FA8702-15-D-0002 with Carnegie Mellon University for the operation of the Software Engineering Institute [DM-0003727, DM19-0929, DM21-0090]
\item Graphegon Inc.
\item Anaconda
\end{itemize}

The following people provided valuable input and feedback during the development of the LAGraph library (in alphabetical order): Benjamin Brock, Ayd\i n Bulu\c{c}, Jos\'e Moreira
\vfill
\pagebreak

%-----------------------------------------------------------------------------

\chapter{Introduction}

General introduction to LAGraph and its dependence on the GraphBLAS.
We need to explain the motivation as well.

Normative standards include GraphBLAS version 2.0 and C99 (ISO/IEC 9899:199) 
extended with {\it static type-based} and {\it number of parameters-based}
function polymorphism, and language extensions on par with 
the {\tt \_Generic} construct from C11 (ISO/IEC 9899:2011).  
Furthermore, the standard assumes programs using the LAGraph Library
will execute on hardware that supports floating point arithmetic
such as that defined by the IEEE~754 (IEEE 754-2008) standard. 

Some more overview text to set the context for what follows

The remainder of this document is organized as follows:
\begin{itemize}
\item Chapter~\ref{Chp:Concepts}: Basic Concepts
\item Chapter~\ref{Chp:Objects}: Objects and defined values
\item Chapter~\ref{Chp:API}: The LAGraph API
\item Appendix~\ref{Chp:RevHistory}: Revision history
\item Appendix~\ref{Chp:Examples}: Examples
\end{itemize}

%=============================================================================
%=============================================================================

\chapter{Basic concepts}
\label{Chp:Concepts}

The LAGraph library is a collection of high level graph algorithms
based on the GraphBLAS C API.  These algorithms construct  
graph algorithms expressed ``in the language of linear algebra.''
Graphs are expressed as matrices, and the operations over 
these matrices are generalized through the use of a
semiring algebraic structure.

In this chapter, we will define the basic concepts used to
define the LAGraph Library  We provide the following elements:

\begin{itemize}
\item Glossary of terms and notation used in this document.  
\item The LAGraph objects. 
\item Return codes and other constants used in LAGraph.
\end{itemize}

Currently, I've kept the text from the GraphBLAS concepts chapter in 
this document.  We may want to borrow some of the GraphBLAS glossary items
and perhaps use some of the table formatting in LAGraph.

\section{Glossary}

%=============================================================================

\subsection{LAGraph basic definitions}

\glossBegin

\glossItem{application} A program that calls methods from the GraphBLAS C API to
solve a problem.

\glossItem{GraphBLAS C API} The application programming interface that fully defines the types, objects, 
literals, and other elements of the C binding to the GraphBLAS.

\glossItem{function} Refers to a named group of statements in the C programming language.  Methods, operators,
and user-defined functions are typically implemented as C functions.
When referring to 
the code programmers write, as opposed to the role of functions as an element of the GraphBLAS, they may
be referred to as such.

\glossItem{method} A function defined in the GraphBLAS C API that manipulates
GraphBLAS objects or other opaque features of the implementation of the GraphBLAS API.

\glossItem{operator} A function that performs an operation on the elements stored in GraphBLAS matrices and vectors.

\glossItem{GraphBLAS operation} A mathematical operation defined in the
GraphBLAS mathematical specification. These operations (not to be confused with \emph{operators}) typically act
on matrices and vectors with elements defined in terms of an algebraic semiring. 
\glossEnd

%=============================================================================

\subsection{LAGraph objects and their structure}

\glossBegin
\glossItem{non-opaque datatype} Any datatype that exposes its internal structure and
can be manipulated directly by the user.   

\glossItem{opaque datatype} Any datatype that hides its internal structure and can
be manipulated only through an API.

\glossItem{GraphBLAS object}  An instance of an \emph{opaque datatype} defined 
by the \emph{GraphBLAS C API} that is manipulated only through the GraphBLAS 
API. There are four kinds of GraphBLAS opaque objects: \emph{domains} (i.e., types), 
\emph{algebraic objects} (operators, monoids and semirings), 
\emph{collections} (scalars, vectors, matrices and masks), and descriptors.   

\glossItem{handle}  A variable that holds a reference to an instance of one of 
the GraphBLAS opaque objects.  The value of this variable holds a reference to 
a GraphBLAS object but not the contents of the object itself.  Hence, assigning 
a value to another variable copies the reference to the GraphBLAS object of one 
handle but not the contents of the object.

\glossItem{domain} The set of valid values for the elements stored in a 
GraphBLAS \emph{collection} or operated on by a GraphBLAS \emph{operator}.
Note that some GraphBLAS objects involve functions that map values from 
one or more input domains onto values in an output domain.  These GraphBLAS 
objects would have multiple domains.

\glossItem{collection} An opaque GraphBLAS object that holds a number of
elements from a specified \emph{domain}. Because these objects are based on an 
opaque datatype, an implementation of the GraphBLAS C API has the flexibility 
to optimize the data structures for a particular platform.  GraphBLAS objects 
are often implemented as sparse data structures, meaning only the subset of the
elements that have values are stored.

\glossItem{implied zero}  Any element that has a valid index (or indices) 
in a GraphBLAS vector or matrix but is not explicitly identified in the list of 
elements of that vector or matrix. From a mathematical perspective, an
\emph{implied zero} is treated as having the 
value of the zero element of the relevant monoid or semiring.
However, GraphBLAS operations are purposefully defined using set notation in such a way
that it makes it unnecessary to reason about implied zeros. 
Therefore, this concept is not used in the definition of GraphBLAS methods and operators.

\glossItem{mask} An internal GraphBLAS object used to control how values 
are stored in a method's output object.  The mask exists only inside a method; hence,
it is called an \emph{internal opaque object}.  A mask is formed from the elements of
a collection object (vector or matrix) input as a mask parameter to a method. GraphBLAS 
allows two types of masks:
\begin{enumerate}
\item In the default 
case, an element of the mask exists for each element that exists in the 
input collection object when the value of that element, when cast to a Boolean type, evaluates to 
{\tt true}.  
\item In the {\it structure only} case, masks have structure but no values. 
The input collection describes a structure whereby an 
element of the mask exists for each element stored in the input collection regardless of its value.
\end{enumerate}

\glossItem{complement} The \emph{complement} of a 
GraphBLAS mask, $M$, is another mask, $M'$, where the elements of $M'$
are those elements from $M$ that \emph{do not} exist.  
\glossEnd

%=============================================================================

\subsection{Algebraic structures used in the GraphBLAS}

\glossBegin
\glossItem{associative operator} In an expression where a binary operator is used 
two or more times consecutively, that operator is \emph{associative} if the result 
does not change regardless of the way operations are grouped (without changing their order). 
In other words, in a sequence of binary operations using the same associative 
operator, the legal placement of parenthesis does not change the value resulting 
from the sequence operations.  Operators that are associative over infinitely 
precise numbers (e.g., real numbers) are not strictly associative when applied to 
numbers with finite precision (e.g., floating point numbers). Such non-associativity 
results, for example, from roundoff errors or from the fact some numbers can not 
be represented exactly as floating point numbers.   In the GraphBLAS specification, 
as is common practice in computing, we refer to operators as \emph{associative} 
when their mathematical definition over infinitely precise numbers is associative 
even when they are only approximately associative when applied to finite precision 
numbers.

No GraphBLAS method will imply a predefined grouping over any associative operators. 
Implementations of the GraphBLAS are encouraged to exploit associativity to optimize 
performance of any GraphBLAS method with this requirement. This holds even if the 
definition of the GraphBLAS method implies a fixed order for the associative operations.

\glossItem{commutative operator} In an expression where a binary operator is used (usually
two or more times consecutively), that operator is \emph{commutative} if the result does 
not change regardless of the order the inputs are operated on.

No GraphBLAS method will imply a predefined ordering over any commutative operators. 
Implementations of the GraphBLAS are encouraged to exploit commutativity to optimize 
performance of any GraphBLAS method with this requirement. This holds even if the 
definition of the GraphBLAS method implies a fixed order for the commutative operations.

\glossItem{GraphBLAS operators} Binary or unary operators that act on elements of GraphBLAS 
objects.  \emph{GraphBLAS operators} are used to express algebraic structures used in the 
GraphBLAS such as monoids and semirings. They are also used as arguments to several
GraphBLAS methods. There are two types of \emph{GraphBLAS operators}: 
(1) predefined operators found in Table~\ref{Tab:PredefOperators} and (2) user-defined 
operators created using {\sf GrB\_UnaryOp\_new()} or {\sf GrB\_BinaryOp\_new()}.

\glossItem{monoid} An algebraic structure consisting of one domain, an associative 
binary operator, and the identity of that operator.  There are two types 
of GraphBLAS monoids: (1) predefined monoids found in 
Table~\ref{Tab:PredefinedMonoids} and (2) user-defined monoids created using . 

\glossItem{semiring} An algebraic structure consisting of a set of allowed values
(the \emph{domain}), a commutative and associative binary operator called addition, a binary operator 
called multiplication (where multiplication distributes over addition),
and identities over addition (\emph{0}) and multiplication (\emph{1}).  The additive
identity is an annihilator over multiplication.   

\glossItem{GraphBLAS semiring} is allowed to diverge from the mathematically 
rigorous definition of a \emph{semiring} since certain combinations of domains, operators, and identity 
elements are useful in graph algorithms even when they do not strictly match the mathematical
definition of a semiring.
There are two types 
of \emph{GraphBLAS semirings}: (1) predefined semirings found in 
Tables~\ref{Tab:PredefinedTrueSemirings} and~\ref{Tab:PredefinedUsefulSemirings}, and (2) user-defined semirings created using 
{\sf GrB\_Semiring\_new()} (see Section~\ref{Sec:AlgebraMethods}).

\glossItem{index unary operator} A variation of the unary operator that operates
on elements of GraphBLAS vectors and matrices along with the index values 
representing their location in the objects.  There are predefined index unary
operators found in Table~\ref{Tab:PredefIndexOperators}), and user-defined
operators created using {\sf GrB\_IndexUnaryOp\_new} (see Section~\ref{Sec:AlgebraMethods}).
\glossEnd

%=============================================================================

\subsection{The execution of an application using the GraphBLAS C API}

\glossBegin
\glossItem{program order} The order of the GraphBLAS method calls in a
thread, as defined by the text of the program.

\glossItem{host programming environment} The GraphBLAS specification defines an API.  
The functions from the API appear in a program.  This program is written using a programming language
and execution environment defined outside of the GraphBLAS.  We refer to this programming environment
as the ``host programming environment''.

\glossItem{execution time} time expended while executing instructions defined by a program.
This term is specifically used in this specification in the context of computations 
carried out on behalf of a call to a GraphBLAS method.

%% The original definition was too narrow by only referencing GraphBLAS methods.
%% Also, I didn't see the reason why the comment about execution time was retrieved was included.
%\glossItem{execution time} The time it takes to execute a GraphBLAS
%method call. Implementations are free, but not mandated, to specify how
%the execution time of a method call can be retrieved.

\glossItem{sequence} A GraphBLAS application uniquely defines a directed
acyclic graph (DAG) of GraphBLAS method calls based on their program order.  
At any point in a program, the state of any GraphBLAS object is defined by a 
subgraph of that DAG.  An ordered collection of GraphBLAS method calls in program order that
defines that subgraph for a particular object is the \emph{sequence} for that object.

\glossItem{complete}  A GraphBLAS object is complete when it can be used in a happens-before relationship
with a method call that reads the variable on another thread.  This concept is used
when reasoning about memory orders in multithreaded programs.  A GraphBLAS object defined on one thread 
that is complete can be safely used as an {\sf IN} or {\sf INOUT} argument
in a method-call on a second thread assuming the method calls are correctly synchronized so the definition on 
the first thread \emph{happens-before} it is used on the second thread.  In blocking-mode, an object is 
complete after a GraphBLAS method call that writes to that object returns.   In nonblocking-mode, an object is complete 
after a call to the {\sf GrB\_wait()} method with the {\sf GrB\_COMPLETE} parameter.

\glossItem{materialize} A GraphBLAS object is materialized when it is (1) complete, (2) the computations 
defined by the sequence that define the object have finished (either fully or stopped at an error) and will not consume any 
additional computational resources, and (3) any errors associated with that sequence are available to be read according to the 
GraphBLAS error model.  A GraphBLAS object that is never loaded into a non-opaque data structure may 
potentially never be materialized.  This might happen, for example, if the operations 
associated with the object are fused or otherwise changed by the runtime system 
that supports the implementation of the GraphBLAS C API.   An object can be materialized by a call
to the materialize mode of the {\sf GrB\_wait()} method. 

\glossItem{context}  An instance of the GraphBLAS C API implementation
as seen by an application.  An application can have only one context between the 
start and end of the application.  
A context begins with the first thread that calls {\sf GrB\_init()} and ends with the 
first thread to call {\sf GrB\_finalize()}.  
It is an error for {\sf GrB\_init()} or {\sf GrB\_finalize()} to be called more than one
time within an application.  The context is used to constrain the behavior of an
instance of the GraphBLAS C API implementation and support various execution strategies.
Currently, the only
supported constraints on a context pertain to the mode of program execution.

\glossItem{program execution mode} Defines how a GraphBLAS sequence executes, and is associated 
with the {\it context} of a GraphBLAS C API implementation. It is set by an 
application with its call to {\sf GrB\_init()} to one of two possible states.  
In \emph{blocking mode}, GraphBLAS methods return after the computations 
complete and any output objects have been materialized.  In {\it nonblocking mode}, a 
method may return once the arguments are tested as consistent with 
the method (\ie, there are no API errors), and potentially before any computation 
has taken place.
\glossEnd

%=============================================================================

\subsection{GraphBLAS methods: behaviors and error conditions}
\glossBegin
\glossItem{implementation-defined behavior} Behavior that must be documented
by the implementation and is allowed to vary among different
compliant implementations. 

\glossItem{undefined behavior} Behavior that is not specified by the GraphBLAS C API.
A conforming implementation is free to choose results delivered from a method
whose behavior is undefined. 

\glossItem{thread-safe}  Consider a function called from multiple threads with 
arguments that do not overlap in memory (i.e. the argument lists do not share 
memory).  If the function is \emph{thread-safe} then it will behave the same 
when executed concurrently by multiple threads or sequentially on a single 
thread.

\glossItem{dimension compatible} GraphBLAS objects (matrices and vectors) that are
passed as parameters to a GraphBLAS method are dimension (or shape) compatible if
they have the correct number of dimensions and sizes for each dimension to satisfy 
the rules of the mathematical definition of the operation associated with the method. 
If any \emph{dimension compatibility} rule above is violated, execution of the GraphBLAS 
method ends and the {\sf GrB\_DIMENSION\_MISMATCH} error is returned.

\glossItem{domain compatible} Two domains for which values from one domain can be 
cast to values in the other domain as per the rules of the C language. In particular, 
domains from Table~\ref{Tab:PredefinedTypes} 
are all compatible with each other, and a domain from a user-defined type is only 
compatible with itself. If any \emph{domain compatibility} rule above is 
violated, execution of the GraphBLAS method ends and the {\sf GrB\_DOMAIN\_MISMATCH} 
error is returned.
\glossEnd

\vfill

\newgeometry{left=2.5cm,top=2cm,bottom=2cm}

%=============================================================================
%=============================================================================

\section{Notation}

\begin{tabular}[H]{l|p{5in}}
Notation & Description \\
\hline
$\Dout, \Dinn, \Din1, \Din2$  & Refers to output and input domains of various GraphBLAS operators. \\
$\bDout(*), \bDinn(*),$ & Evaluates to output and input domains of GraphBLAS operators (usually \\
~~~~$\bDin1(*), \bDin2(*)$ & a unary or binary operator, or semiring). \\
$\mathbf{D}(*)$   & Evaluates to the (only) domain of a GraphBLAS object (usually a monoid, vector, or matrix). \\ 
$f$             & An arbitrary unary function, usually a component of a unary operator. \\
$\mathbf{f}(F_u)$ & Evaluates to the unary function contained in the unary operator given as the argument. \\
$\odot$         & An arbitrary binary function, usually a component of a binary operator. \\
$\mathbf{\bigodot}(*)$ & Evaluates to the binary function contained in the binary operator or monoid given as the argument. \\
$\otimes$       & Multiplicative binary operator of a semiring. \\
$\oplus$        & Additive binary operator of a semiring. \\
$\mathbf{\bigotimes}(S)$ & Evaluates to the multiplicative binary operator of the semiring given as the argument. \\
$\mathbf{\bigoplus}(S)$ & Evaluates to the additive binary operator of the semiring given as the argument. \\
$\mathbf{0}(*)$   & The identity of a monoid, or the additive identity of a GraphBLAS semiring. \\
$\mathbf{L}(*)$   & The contents (all stored values) of the vector or matrix GraphBLAS objects.  For a vector, it is the set of (index, value) pairs, and for a matrix it is the set of (row, col, value) triples. \\
$\mathbf{v}(i)$ or $v_i$   & The $i^{th}$ element of the vector $\vector{v}$.\\
$\mathbf{size}(\vector{v})$ & The size of the vector $\vector{v}$.\\
$\mathbf{ind}(\vector{v})$ & The set of indices corresponding to the stored values of the vector $\vector{v}$.\\
$\mathbf{nrows}(\vector{A})$ & The number of rows in the $\matrix{A}$.\\
$\mathbf{ncols}(\vector{A})$ & The number of columns in the $\matrix{A}$.\\
$\mathbf{indrow}(\vector{A})$ & The set of row indices corresponding to rows in $\matrix{A}$ that have stored values.  \\
$\mathbf{indcol}(\vector{A})$ & The set of column indices corresponding to columns in $\matrix{A}$ that have stored values. \\
$\mathbf{ind}(\vector{A})$ & The set of $(i,j)$ indices corresponding to the stored values of the matrix. \\
$\mathbf{A}(i,j)$ or $A_{ij}$ & The element of $\matrix{A}$ with row index $i$ and column index $j$.\\
$\matrix{A}(:,j)$ & The $j^{th}$ column of matrix $\matrix{A}$.\\
$\matrix{A}(i,:)$ & The $i^{th}$ row of matrix $\matrix{A}$.\\
$\matrix{A}^T$ &The transpose of matrix $\matrix{A}$. \\
$\neg\matrix{M}$ & The complement of $\matrix{M}$.\\
s$(\matrix{M})$ & The structure of $\matrix{M}$.\\
$\vector{\widetilde{t}}$ & A temporary object created  by the GraphBLAS implementation. \\
$<type>$ & A method argument type that is {\sf void *} or one of the types from Table~\ref{Tab:PredefinedTypes}. \\
{\sf GrB\_ALL} & A method argument literal to indicate that all indices of an input array should be used.\\
{\sf GrB\_Type} & A method argument type that is either a user defined type or one of the  types from Table~\ref{Tab:PredefinedTypes}.\\
{\sf GrB\_Object} &  A method argument type referencing any of the GraphBLAS object types.\\
{\sf GrB\_NULL} & The GraphBLAS NULL.
\end{tabular}

\restoregeometry


\section{Mathematical foundations}

Graphs can be represented in terms of matrices. The values stored in these 
matrices correspond to attributes (often weights) of edges in the graph.\footnote{More information on the mathematical foundations can be found in the following paper: J. Kepner, P. Aaltonen, D. Bader,  A. Buluç, F. Franchetti, J. Gilbert, D. Hutchison, M. Kumar, A. Lumsdaine, H. Meyerhenke, S. McMillan, J. Moreira, J. Owens, C. Yang, M. Zalewski, and T. Mattson. 2016, September. Mathematical foundations of the GraphBLAS. In \emph{2016 IEEE High Performance Extreme Computing Conference (HPEC)} (pp. 1-9). IEEE.} 
Likewise, information about vertices in a graph are stored in vectors.
The set of valid values that can be stored in either matrices or vectors
is referred to as their domain. Matrices are usually sparse because the 
lack of an edge between two vertices means that nothing is stored at the 
corresponding location in the matrix.  Vectors may be sparse or dense, or they may 
start out sparse and become dense as algorithms traverse the graphs.

Operations defined by the GraphBLAS C API specification operate on these 
matrices and vectors to carry out graph algorithms.  These GraphBLAS 
operations are defined in terms of GraphBLAS semiring algebraic 
structures. Modifying the underlying semiring changes the result of 
an operation to support a wide range of graph algorithms.
Inside a given algorithm, it is often beneficial to change the GraphBLAS 
semiring that applies to an operation on a matrix.  This has two 
implications for the C binding of the GraphBLAS API.  

First, it means that we define a separate object for the semiring 
to pass into methods.  Since in many cases the full
semiring is not required, we also support passing monoids or
even binary operators, which means the semiring is implied rather than 
explicitly stated.

Second, the ability to change semirings impacts the meaning of 
the \emph{implied zero} in a sparse representation of a matrix or vector.
This element in real arithmetic is zero, which is the 
identity of the \emph{addition} operator and the annihilator of the
\emph{multiplication} operator.  As the semiring changes, this 
implied zero changes to the identity of the \emph{addition} operator 
and the annihilator (if present) of the \emph{multiplication} operator 
for the new semiring. Nothing changes regarding what is stored in the sparse 
matrix or vector, but the implied zeros within them change with respect to a 
particular operation. In all cases, the nature of the implied zero does not 
matter since the GraphBLAS C API requires that implementations treat them as 
nonexistent elements of the matrix or vector.

As with matrices and vectors, GraphBLAS semirings have domains
associated with their inputs and outputs.  The semirings in the 
GraphBLAS C API are defined with two domains associated with the input operands and one 
domain associated with output.  When used in the GraphBLAS C API these
domains may not match the domains of the matrices and vectors supplied in
the operations.  In this case, only valid \emph{domain compatible} casting 
is supported by the API.

The mathematical formalism for graph operations in the language of 
linear algebra often assumes that we can operate in the field of real numbers. 
However, the GraphBLAS C binding is designed for implementation on computers, 
which by necessity have a finite number of bits to represent numbers. 
Therefore, we require a conforming implementation to use floating point 
numbers such as those defined by the IEEE-754 standard (both single- and double-precision) 
wherever real numbers need to be represented. The practical implications of 
these finite precision numbers is that the result of a sequence of 
computations may vary from one execution to the next as the grouping of operands
(because of associativity) within the operations changes.  While techniques are known to 
reduce these effects, we do not require or even expect an implementation 
to use them as they may add considerable overhead. In most 
cases, these roundoff errors are not significant. When they are significant, 
the problem itself is ill-conditioned and needs to be reformulated.


\section{LAgraph  objects}

Objects defined in the GraphBLAS standard include types (the domains of 
elements), collections of elements (matrices, vectors, and scalars), operators 
on those elements (unary, index unary, and binary operators), algebraic 
structures (semirings and monoids), and descriptors.   GraphBLAS objects are 
defined as opaque types; that is, they are managed, manipulated, and accessed 
solely through the GraphBLAS application programming interface. This gives an 
implementation of the GraphBLAS C specification flexibility to optimize objects 
for different scenarios or to meet the needs of different hardware platforms.

A GraphBLAS opaque object is accessed through its \emph{handle}.  A handle is 
a variable that references an instance of one of the types from 
Table~\ref{Tab:ObjTypes}.  An implementation of the GraphBLAS specification 
has a great deal of flexibility in how these handles are implemented.  All 
that is required is that the handle corresponds to a type defined in the 
C language that supports assignment and comparison for equality.  The
GraphBLAS specification defines a literal {\sf GrB\_INVALID\_HANDLE} that is 
valid for each type.  Using the logical equality operator from C, it must be 
possible to compare a handle to {\sf GrB\_INVALID\_HANDLE} to verify that a 
handle is valid.


\begin{table}
\hrule
\begin{center}
\caption{Types of GraphBLAS opaque objects.}
\label{Tab:ObjTypes}
~\\
\begin{tabular}{l|l}
{\sf GrB\_Object types} & Description \\
\hline
{\sf GrB\_Type}           & Scalar type.     \\ \hline
{\sf GrB\_UnaryOp}        & Unary operator.     \\
{\sf GrB\_IndexUnaryOp}   & Unary operator, that operates on a single value and its location index values.     \\
{\sf GrB\_BinaryOp}       & Binary operator.     \\
{\sf GrB\_Monoid}         & Monoid algebraic structure.     \\
{\sf GrB\_Semiring}       & A GraphBLAS semiring algebraic structure. \\ \hline
{\sf GrB\_Scalar}         & One element; could be empty. \\ 
{\sf GrB\_Vector}         & One-dimensional collection of elements; can be sparse.     \\
{\sf GrB\_Matrix}         & Two-dimensional collection of elements; typically sparse.    \\ \hline
{\sf GrB\_Descriptor}     & Descriptor object, used to modify behavior of methods (specifically \\
                          & GraphBLAS operations).     \\
\end{tabular}
\end{center}
\hrule
\end{table}

Every GraphBLAS object has a \emph{lifetime}, which consists of
the sequence of instructions executed in program order between the
\emph{creation} and the \emph{destruction} of the object.  The GraphBLAS C
API predefines a number of these objects which are created
when the GraphBLAS context is initialized by a call to {\sf GrB\_init}
and are destroyed when the GraphBLAS context is terminated by a call to
{\sf GrB\_finalize}.

An application using the GraphBLAS API can create additional objects by
declaring variables of the appropriate type from Table~\ref{Tab:ObjTypes} for 
the objects it will use.  Before use, the object must be initialized 
with a call call to one of the object's respective \emph{constructor} methods.  
Each kind of object has at least one explicit constructor method of the form 
{\sf GrB\_*\_new} where `{\sf *}' is replaced with the type of object (e.g., 
{\sf GrB\_Semiring\_new}). Note that some objects, especially collections, 
have additional constructor methods such as duplication, import, or 
deserialization.  Objects explicitly created by a call to a constructor 
should be destroyed by a call to {\sf GrB\_free}. The behavior of a program
that calls {\sf GrB\_free} on a pre-defined object is undefined.

%This is typically done with one of 
%the methods that has a ``{\sf \_new}'' suffix in its name (e.g., 
%{\sf GrB\_Vector\_new}).  If available, an object can also be initialized by 
%duplicating an existing object with one of the methods that has the 
%``{\sf \_dup}'' suffix in its name  (e.g., {\sf GrB\_Vector\_dup}).  Note that 
%there are other valid constructor methods included in the API (e.g., 
%``{\sf \_diag}'', ``{\sf \_import}'', and ``{\sf \_deserialize}'' matrix 
%methods).  Regardless of the method of construction, any resources associated 
%with that object can be released (destructed) by a call to the {\sf GrB\_free} 
%method when an application is finished with an object.    

These constructor and destructor methods are the only methods that change 
the value of a handle.  Hence, objects changed by these methods are passed
into the method as pointers.  In all other cases, handles are not changed by the 
method and are passed by value.  For example, even when multiplying matrices, 
while the contents of the output product matrix changes, the handle for that 
matrix is unchanged. 

Several GraphBLAS constructor methods take other objects as input arguments
and use these objects to create a new object. For all these
methods, the lifetime of the created object must end strictly before
the lifetime of any dependent input objects. For example, a vector constructor
{\sf GrB\_Vector\_new} takes a {\sf GrB\_Type} object as input. That type
object must not be destroyed until after the created vector is destroyed.
Similarly, a {\sf GrB\_Semiring\_new} method takes a monoid and
a binary operator as inputs. Neither of these can be destroyed until
after the created semiring is destroyed.

Note that some constructor methods like {\sf GrB\_Vector\_dup} and 
{\sf GrB\_Matrix\_dup} behave differently. In these cases, the input 
vector or matrix can
be destroyed as soon as the call returns. However, the original type
object used to create the input vector or matrix cannot be destroyed
until after the vector or matrix created by {\sf GrB\_Vector\_dup} or
{\sf GrB\_Matrix\_dup} is destroyed.  This behavior must hold for any
chain of duplicating constructors.

Programmers using GraphBLAS handles must be careful to distinguish between a 
handle and the object manipulated through a handle.  For example, a program may 
declare two GraphBLAS objects of the same type, initialize one, and then assign 
it to the other variable.  That assignment, however, only assigns the handle to 
the variable.  It does not create a copy of that variable (to do that, one 
would need to use the appropriate duplication method).  If later the object is 
freed by calling {\sf GrB\_free} with the first variable, the object is 
destroyed and the second variable is left referencing an object that no longer 
exists (a so-called ``dangling handle'').

In addition to opaque objects manipulated through handles, the GraphBLAS C API 
defines an additional opaque object as an internal object; that is, the object 
is never exposed as a variable within an application.  This opaque object is 
the mask used to control which computed values can be stored in the output 
operand of a \emph{GraphBLAS operation}.  .


\chapter{Objects}
\label{Chp:Objects}

In this chapter, all of the enumerations, literals, data types, and predefined 
opaque objects defined in the GraphBLAS API are presented.  Enumeration literals 
in GraphBLAS are assigned specific values to ensure compatibility between 
different runtime library implementations.  The chapter starts by defining the
enumerations that are used by the {\sf init()} and {\sf wait()} methods.  Then
a number of transparent (i.e., non-opaque) types that are used for interfacing 
with external data are defined.  Sections that follow describe the various
types of opaque objects in GraphBLAS: types (or \emph{domains}), algebraic 
objects, collections and descriptors.  Each of these sections also lists the 
predefined instances of each opaque type that are required by the API.  This chapter
concludes with a section on the definition for {\sf GrB\_Info} enumeration 
that is used as the return type of all methods.

%============================================================================
\section{Enumerations for {\sf init()} and {\sf wait()}}

Table~\ref{Tab:EnumerationModes} lists the enumerations and the corresponding
values used in the {\sf GrB\_init()} method to set the execution mode and in
the {\sf GrB\_wait()} method for completing or materializing opaque objects.

%============================================================================
\section{Indices, index arrays, and scalar arrays}

In order to interface with third-party software (\ie, software other than
an implementation of the GraphBLAS), operations 
such as {\sf GrB\_Matrix\_build} (Section~\ref{Sec:Matrix_build}) and
{\sf GrB\_Matrix\_extractTuples} (Section~\ref{Sec:Matrix_extractTuples}) must specify
how the data should be laid out in  non-opaque data structures.  To 
this end we explicitly define the types for indices and the arrays 
used by these operations.

For indices a {\sf typedef} is used to give a GraphBLAS name to a concrete type. We define it as follows:
\begin{verbatim}
    typedef uint64_t GrB_Index;
\end{verbatim}
The range of valid values for a variable of type {\sf GrB\_Index} is [0, {\sf GrB\_INDEX\_MAX}] 
where the largest index value permissible is defined with a macro, {\sf GrB\_INDEX\_MAX}. For example:
\begin{verbatim}
    #define GrB_INDEX_MAX ((GrB_Index) 0x0fffffffffffffff);
\end{verbatim}
An implementation is required to define and document this value.

An index array is a pointer to a set of {\sf GrB\_Index} values that are 
stored in a contiguous block of memory (\ie, {\sf GrB\_Index*}).
Likewise, a scalar array is a pointer to a contiguous block of memory 
storing a number of scalar values as specified by the user.
Some GraphBLAS operations (\eg, {\sf GrB\_assign})  include an input parameter with the type of an index array. 
This input index array selects a subset of elements from a GraphBLAS vector or matrix object to be used in the operation.
In these cases, the literal {\sf GrB\_ALL} 
can be used in place of the index array input parameter to indicate that all indices 
of the associated GraphBLAS vector or matrix object should be used.
%As with any literal defined in the GraphBLAS, 
An implementation of the GraphBLAS C API has considerable 
freedom in terms of how {\sf GrB\_ALL} is defined.  Since {\sf GrB\_ALL} is used as an argument for an array 
parameter, it must use a type consistent with a pointer. {\sf GrB\_ALL} must also have a non-null
value to distinguish it from the erroneous case of passing a {\sf NULL} pointer as an array.

\begin{table}[b!]
\hrule
\begin{center}
\caption{Enumeration literals and corresponding values input to various GraphBLAS methods.}
\label{Tab:EnumerationModes}

\vspace{1\baselineskip}
(a) {\sf GrB\_Mode} execution modes for the {\sf GrB\_init} method.
\vspace{1\baselineskip}

\begin{tabular}{l|r|p{4in}}
Symbol    & Value & Description \\ \hline
{\sf GrB\_NONBLOCKING}   &  0 & Specifies the nonblocking mode context.\\
{\sf GrB\_BLOCKING}      &  1 & Specifies the blocking mode context. \\
\end{tabular}

\vspace{2\baselineskip}
(b) {\sf GrB\_WaitMode} wait modes for the {\sf GrB\_wait} method.
\vspace{1\baselineskip}

\begin{tabular}{l|r|p{4in}}
Symbol    & Value & Description \\ \hline
{\sf GrB\_COMPLETE}    &  0 & The object is in a state where it can be used in a happens-before relation so that multithreaded programs can be properly synchronized.\\
{\sf GrB\_MATERIALIZE} &  1 & The object is \emph{complete}, and in addition, all computation of the object is finished and any error information is available. \\
\end{tabular}

\end{center}
\hrule
\end{table}

%============================================================================
\section{Types (domains)}
\label{Sec:Domains}

In GraphBLAS, domains correspond to the valid values for types from the
host language (in our case, the C programming language).  GraphBLAS defines
a number of operators that take elements from one or more domains and produce elements of a (possibly) different domain.  GraphBLAS also defines 
three kinds of collections: matrices, vectors and scalars.  For any given 
collection, the elements of the collection belong to a \emph{domain}, which 
is the set of valid values for the elements.  For any variable 
or object $V$ in GraphBLAS we denote as $\mathbf{D}(V)$ the domain of $V$,
that is, the set of possible values that elements of $V$ can take.  

The domains for elements that can be stored in collections and operated on
through GraphBLAS methods are defined by GraphBLAS objects called {\sf GrB\_Type}.
The predefined types and corresponding domains used in the GraphBLAS C API are
shown in Table~\ref{Tab:PredefinedTypes}.  The Boolean type ({\tt bool})
is defined in {\tt stdbool.h}, the integral types ({\tt int8\_t},
{\tt uint8\_t}, {\tt int16\_t}, {\tt uint16\_t}, {\tt int32\_t},
{\tt uint32\_t}, {\tt int64\_t}, {\tt uint64\_t}) are defined in {\tt
stdint.h}, and the floating-point types ({\tt float}, {\tt double}) are
native to the language and platform and in most cases defined by the 
IEEE-754 standard.

\begin{table}
\hrule
\begin{center}
\caption[Predefined {\sf GrB\_Type} values.]{Predefined {\sf GrB\_Type} values, and the corresponding GraphBLAS domain 
suffixes, C type (for scalar parameters), and domains for GraphBLAS.  The domain
suffixes are used in place of $I$, $F$, and $T$ in 
Tables~\ref{Tab:PredefOperators}, \ref{Tab:PredefIndexOperators}, 
\ref{Tab:PredefinedMonoids}, \ref{Tab:PredefinedTrueSemirings}, 
and~\ref{Tab:PredefinedUsefulSemirings}).}
\label{Tab:PredefinedTypes}
\label{Tab:PredefinedDomains}

\vspace{1\baselineskip}
\begin{tabular}{l|l|l|l}
{\sf GrB\_Type}   & Suffix       & C type          & Domain \\
\hline
{\sf GrB\_BOOL}   & {\sf BOOL}   & {\tt bool}      & $\{ {\tt false}, {\tt true} \}$  \\
{\sf GrB\_INT8}   & {\sf INT8}   & {\tt int8\_t}   & $\mathbb{Z} \cap [-2^{7},2^{7})$  \\
{\sf GrB\_UINT8}  & {\sf UINT8}  & {\tt uint8\_t}  & $\mathbb{Z} \cap [0,2{^8})$  \\
{\sf GrB\_INT16}  & {\sf INT16}  & {\tt int16\_t}  & $\mathbb{Z} \cap [-2^{15},2^{15})$ \\
{\sf GrB\_UINT16} & {\sf UINT16} & {\tt uint16\_t} & $\mathbb{Z} \cap [0,2^{16})$ \\
{\sf GrB\_INT32}  & {\sf INT32}  & {\tt int32\_t}  & $\mathbb{Z} \cap [-2^{31},2^{31})$ \\
{\sf GrB\_UINT32} & {\sf UINT32} & {\tt uint32\_t} & $\mathbb{Z} \cap [0,2^{32})$ \\
{\sf GrB\_INT64}  & {\sf INT64}  & {\tt int64\_t}  & $\mathbb{Z} \cap [-2^{63},2^{63})$ \\
{\sf GrB\_UINT64} & {\sf UINT64} & {\tt uint64\_t} & $\mathbb{Z} \cap [0,2^{64})$ \\
{\sf GrB\_FP32}   & {\sf FP32}   & {\tt float}     & IEEE 754 {\sf binary32}  \\
{\sf GrB\_FP64}   & {\sf FP64}   & {\tt double}    & IEEE 754 {\sf binary64}  
\end{tabular}
\end{center}
\hrule
\end{table}

%============================================================================
\section{Algebraic objects, operators and associated functions}

GraphBLAS operators operate on elements stored in GraphBLAS collections. A 
\emph{binary operator} is a function that maps two input values to one 
output value. A \emph{unary operator} is a function that maps one input value 
to one output value.  Binary operators are defined over two input domains
and produce an output from a (possibly different) third domain. Unary
operators are specified over one input domain and produce an output from a
(possibly different) second domain.

In addition to the operators that operate on stored values, GraphBLAS
also supports \emph{index unary operators} that maps a stored value and 
the indices of its position in the matrix or vector to an output value.
That output value can be used in the index unary operator variants of {\sf apply} (\S~\ref{Sec:Apply}) 
to compute a new stored value, or be used in the {\sf select} operation (\S~\ref{Sec:Select}) to 
determine if the stored input value should be kept or annihilated.

Some GraphBLAS operations require a monoid or semiring.  A monoid contains an associative 
binary operator where the input and output domains are
the same. The monoid also includes an identity value of the operator.
The semiring consists of a binary operator -- referred to as the ``times'' 
operator -- with up to three different domains (two inputs
and one output) and a monoid -- referred to as the ``plus'' operator -- that
is also commutative.  Furthermore, the domain
of the monoid must be the same as the output domain of the ``times'' operator.

The GraphBLAS \emph{algebraic objects} operators, monoids, and semirings
are presented in this section.
These objects can be used as input arguments to various GraphBLAS
operations, as shown in Table~\ref{Tab:OperatorInputType}.
The specific rules for each algebraic object
are explained in the respective sections of those objects.  A summary
of the properties and recipes for building these GraphBLAS algebraic
objects is presented in Table~\ref{Tab:AlgebraicObjects}.

\begin{table}[t]
    \hrule
    \begin{center}
        \caption[Operator input for relevant GraphBLAS operations.]{Operator input for relevant GraphBLAS operations. 
        The semiring add and times are shown if applicable.}
        \label{Tab:OperatorInputType}
        \begin{tabular}{l|l}
        Operation                       & Operator input        \\ \hline
        {\sf mxm, mxv, vxm}             & semiring              \\ \hline
        {\sf eWiseAdd}                  & binary operator       \\
                                        & monoid                \\
                                        & semiring (add)        \\ \hline
        {\sf eWiseMult}                 & binary operator       \\
                                        & monoid                \\
                                        & semiring (times)      \\ \hline
       {\sf reduce} (to vector or {\sf GrB\_Scalar})  & binary operator    \\ 
                                        & monoid                \\ \hline
       {\sf reduce} (to scalar value)   & monoid                \\ \hline
       {\sf apply}                      & unary operator        \\
	                                    & binary operator with scalar \\
                                        & index unary operator  \\ \hline
       {\sf select}                     & index unary operator  \\ \hline
       {\sf kronecker}                  & binary operator       \\
                                        & monoid                \\
                                        & semiring              \\ \hline
       {\sf dup} argument (build methods)     & binary operator \\ \hline
       {\sf accum} argument (various methods) & binary operator \\
       \end{tabular}
    \end{center}
    \hrule
\end{table}

%====================

\begin{table}
    \hrule
    \begin{center}
        \caption[Properties and recipes for building GraphBLAS algebraic objects.]{Properties and recipes for building GraphBLAS algebraic objects: unary operator, binary operator, monoid, and semiring (composed of operations \emph{add} and \emph{times}).}
        \label{Tab:AlgebraicObjects}
        
        \vspace{1\baselineskip}
        (a) Properties of algebraic objects.
        \vspace{1\baselineskip}
        
        \begin{tabular}{l|l|l|l|l}
            Object          & Must be       & Must be        & Identity         & Number \\
                            & commutative   & associative    & must exist       & of domains  \\
            \hline
            Unary operator  & n/a           & n/a            & n/a              & 2  \\
            Binary operator & no            & no             & no               & 3  \\
            Monoid          & no            & yes            & yes              & 1  \\
            Reduction add   & yes           & yes            & yes (see Note 1) & 1  \\
            Semiring add    & yes           & yes            & yes              & 1  \\
            Semiring times  & no            & no             & no               & 3  (see Note 2) \\
        \end{tabular}
        
        \vspace{1\baselineskip}
        (b) Recipes for algebraic objects.
        \vspace{1\baselineskip}
        
        \begin{tabular}{l|l|l}
            Object          & Recipe                                        & Number of domains \\
            \hline
            Unary operator  & Function pointer                              & 2 \\
            Binary operator & Function pointer                              & 3 \\
            Monoid          & Associative binary operator with identity     & 1 \\
            Semiring        & Commutative monoid $+$ binary operator        & 3 \\
        \end{tabular}
        
    \end{center}

        {\footnotesize Note 1: Some high-performance GraphBLAS implementations may require 
        an identity to perform reductions to sparse objects like GraphBLAS vectors 
        and scalars. According to the descriptions of the corresponding GraphBLAS operations, 
        however, this identity is mathematically not necessary.  There are API signatures to
        support both.\newline
        Note 2: The output domain of the semiring times must be same as the domain of the 
        semiring's add monoid. This ensures three domains for a semiring rather than four.}

    \hrule
\end{table}

%====================

A number of predefined operators are specified by the GraphBLAS C API.  They
are presented in tables in their respective subsections below. Each of these 
operators is defined to operate on specific GraphBLAS types and therefore, 
this type is built into the name of the object as a suffix.  These suffixes 
and the corresponding predefined {\sf GrB\_Type} objects that are listed in 
Table~\ref{Tab:PredefinedTypes}.

%----------------------------------------------------------------------------
\subsection{Operators}

A GraphBLAS \emph{unary operator} $F_u = \langle \Dout, \Dinn, f\rangle$
is defined by two domains, $\Dout$ and $\Dinn$, and an operation
$f: \Dinn \rightarrow \Dout$.  For a given GraphBLAS unary operator
$F_u=\langle \Dout, \Dinn, f \rangle$, we define $\bDout(F_u) = \Dout$, 
$\bDinn(F_u) = \Dinn$, and $\mathbf{f}(F_u) = f$.

A GraphBLAS \emph{binary operator} $F_b = \langle \Dout, \Din1, \Din2, 
\odot \rangle$
is defined by three domains, $\Dout$, $\Din1$, $\Din2$, and an operation
$\odot: \Din1 \times \Din2 \rightarrow \Dout$.  For a given GraphBLAS binary operator
$F_b=\langle \Dout, \Din1, \Din2, \odot \rangle$, we define $\bDout(F_b) = \Dout$,
$\bDin1(F_b) = \Din1$, $\bDin2(F_b) = \Din2$, and $\mathbf{\bigodot}(F_b)
= \odot$.  Note that $\odot$ could be used in place of either $\oplus$ or 
$\otimes$ in other methods and operations. 

A GraphBLAS \emph{index unary operator} 
$F_i = \langle \Dout, \Din1, \mathbf{D}({\sf GrB\_Index}), \Din2, f_{i} \rangle$
is defined by three domains, $\Dout$, $\Din1$, $\Din2$, the domain of GraphBLAS 
indices, and an operation
$f_i: \Din1 \times I_{U64}^2 \times \Din2 \rightarrow \Dout$ (where $I_{U64}$ corresponds to the domain of a {\sf GrB\_Index}).  For a given GraphBLAS 
index operator $F_i$, we define $\bDout(F_i) = \Dout$, 
$\bDin1(F_i) = \Din1$, $\bDin2(F_i) = \Din2$, and $\mathbf{f}(F_i) = f_i$.

User-defined operators can be created with calls to {\sf GrB\_UnaryOp\_new}, 
{\sf GrB\_BinaryOp\_new}, and {\sf GrB\_IndexUnaryOp\_new}, respectively.  
See Section~\ref{Sec:AlgebraMethods} for information on these methods.
The GraphBLAS C API predefines a number of these operators.  These are listed 
in Tables~\ref{Tab:PredefOperators} and~\ref{Tab:PredefIndexOperators}.  
Note that most entries in these tables represent a
``family'' of predefined operators for a set of different types represented by
the $T$, $I$, or $F$ in their names.  For example, the multiplicative inverse 
({\sf GrB\_MINV\_$F$}) function is only defined
for floating-point types ($F = $ {\sf FP32} or {\sf FP64}).  The division
({\sf GrB\_DIV\_$T$}) function is defined for all types, but only if $y
\neq 0$ for integral  and floating point types and $y \neq {\tt false}$ for 
the Boolean type.

%====================
\begin{table}
\hspace*{-2.5em}\begin{threeparttable}
\hrule
\caption[Predefined unary and binary operators for GraphBLAS in C.]{Predefined unary and binary operators for GraphBLAS in C.  The $T$ can 
be any suffix from Table~\ref{Tab:PredefinedDomains}, $I$ can be any integer 
suffix from Table~\ref{Tab:PredefinedDomains}, and $F$ can be any floating-point suffix from Table~\ref{Tab:PredefinedDomains}.}
\label{Tab:PredefOperators}
\vspace{1\baselineskip}

\begin{tabular}{l|l|l|ll}
Operator & GraphBLAS             &                                                              & \\
type     & identifier            & Domains                                              & Description \\ \hline
{\sf GrB\_UnaryOp}    & {\sf GrB\_IDENTITY\_$T$} & $T \rightarrow T $     & $f(x) = x$, &identity \\
{\sf GrB\_UnaryOp}    & {\sf GrB\_ABS\_$T$}      & $T \rightarrow T $     & $f(x) = |x|$, &absolute value \\
{\sf GrB\_UnaryOp}    & {\sf GrB\_AINV\_$T$}     & $T \rightarrow T $     & $f(x) = -x$, &additive inverse \\
{\sf GrB\_UnaryOp}    & {\sf GrB\_MINV\_$F$}     & $F \rightarrow F $     & $f(x) = \frac{1}{x}$, &multiplicative inverse \\
{\sf GrB\_UnaryOp}    & {\sf GrB\_LNOT}          & ${\tt bool} \rightarrow {\tt bool}$  & $f(x) =~\neg x$, &logical inverse  \\
{\sf GrB\_UnaryOp}    & {\sf GrB\_BNOT\_$I$}     & $I \rightarrow I$      & $f(x) =~\mbox{\~{}} x$, &bitwise complement \\

&&&\\
{\sf GrB\_BinaryOp}   & {\sf GrB\_LOR}        & ${\tt bool} \times {\tt bool} \rightarrow {\tt bool}$ & $f(x,y) = x \lor y$, & logical OR \\
{\sf GrB\_BinaryOp}   & {\sf GrB\_LAND}       & ${\tt bool} \times {\tt bool} \rightarrow {\tt bool}$ & $f(x,y) = x \land y$, & logical AND \\
{\sf GrB\_BinaryOp}   & {\sf GrB\_LXOR}       & ${\tt bool} \times {\tt bool} \rightarrow {\tt bool}$ & $f(x,y) = x \oplus y$, & logical XOR \\
{\sf GrB\_BinaryOp}   & {\sf GrB\_LXNOR}      & ${\tt bool} \times {\tt bool} \rightarrow {\tt bool}$ & $f(x,y) = \overline{x \oplus y}$, & logical XNOR \\

{\sf GrB\_BinaryOp}   & {\sf GrB\_BOR\_$I$}   & $I \times I \rightarrow I$ & $f(x,y) = x ~|~ y$, & bitwise OR \\
{\sf GrB\_BinaryOp}   & {\sf GrB\_BAND\_$I$}  & $I \times I \rightarrow I$ & $f(x,y) = x ~\&~ y$, & bitwise AND \\
{\sf GrB\_BinaryOp}   & {\sf GrB\_BXOR\_$I$}  & $I \times I \rightarrow I$ & $f(x,y) = x ~\mbox{\^{}}~ y$, & bitwise XOR \\
{\sf GrB\_BinaryOp}   & {\sf GrB\_BXNOR\_$I$} & $I \times I \rightarrow I$ & $f(x,y) = \overline{x ~\mbox{\^{}}~ y}$, & bitwise XNOR \\

{\sf GrB\_BinaryOp}   & {\sf GrB\_EQ\_$T$}    & $T \times T \rightarrow {\tt bool}$  & $f(x,y) = (x == y)$ & equal \\
{\sf GrB\_BinaryOp}   & {\sf GrB\_NE\_$T$}    & $T \times T \rightarrow {\tt bool}$  & $f(x,y) = (x \neq y)$ & not equal \\
{\sf GrB\_BinaryOp}   & {\sf GrB\_GT\_$T$}    & $T \times T \rightarrow {\tt bool}$  & $f(x,y) = (x > y)$ & greater than  \\
{\sf GrB\_BinaryOp}   & {\sf GrB\_LT\_$T$}    & $T \times T \rightarrow {\tt bool}$  & $f(x,y) = (x < y)$ & less than  \\
{\sf GrB\_BinaryOp}   & {\sf GrB\_GE\_$T$}    & $T \times T \rightarrow {\tt bool}$  & $f(x,y) = (x \geq y)$ & greater than or equal \\
{\sf GrB\_BinaryOp}   & {\sf GrB\_LE\_$T$}    & $T \times T \rightarrow {\tt bool}$  & $f(x,y) = (x \leq y)$ & less than or equal \\
{\sf GrB\_BinaryOp}   & {\sf GrB\_ONEB\_$T$}  & $T \times T \rightarrow T$  & $f(x,y) = 1$, & 1 (cast to $T$) \\
{\sf GrB\_BinaryOp}   & {\sf GrB\_FIRST\_$T$} & $T \times T \rightarrow T$  & $f(x,y) = x$, & first argument \\
{\sf GrB\_BinaryOp}   & {\sf GrB\_SECOND\_$T$}& $T \times T \rightarrow T$  & $f(x,y) = y$, & second argument \\
{\sf GrB\_BinaryOp}   & {\sf GrB\_MIN\_$T$}   & $T \times T \rightarrow T$  & $f(x,y) = (x < y)~?~x : y$, & minimum \\
{\sf GrB\_BinaryOp}   & {\sf GrB\_MAX\_$T$}   & $T \times T \rightarrow T$  & $f(x,y) = (x > y)~?~x : y$, & maximum \\
{\sf GrB\_BinaryOp}   & {\sf GrB\_PLUS\_$T$}  & $T \times T \rightarrow T$  & $f(x,y) = x + y$, & addition \\
{\sf GrB\_BinaryOp}   & {\sf GrB\_MINUS\_$T$} & $T \times T \rightarrow T$  & $f(x,y) = x - y$, & subtraction \\
{\sf GrB\_BinaryOp}   & {\sf GrB\_TIMES\_$T$} & $T \times T \rightarrow T$  & $f(x,y) = xy$, & multiplication \\
{\sf GrB\_BinaryOp}   & {\sf GrB\_DIV\_$T$}   & $T \times T \rightarrow T$  & $f(x,y) = \frac{x}{y}$, & division \\
\end{tabular}
\hrule
\comment{
{\sf GrB\_BinaryOp}   & {\sf GrB\_ANY\_$T$}   & $T \times T \rightarrow T$  & $f(x,y) = $ either $x$ or $y$, & either input operand\tnote{1} \\
\begin{tablenotes}
    \item[1] For {\sf GrB\_ANY}, an implementation is free to return either input operand, and is not required to always return the same operand in different invocations.
\end{tablenotes}
}
\end{threeparttable}
\end{table}

%==================
\begin{landscape}

\begin{table}
\hspace{-2.5em}\begin{threeparttable}
\hrule
%\vspace{1\baselineskip}
\caption[Predefined index unary operators for GraphBLAS in C.]{Predefined index unary operators for GraphBLAS in C.  The $T$ can be
any suffix from Table~\ref{Tab:PredefinedDomains}. $I_{U64}$ refers to the 
    unsigned 64-bit, {\sf GrB\_Index}, integer type, $I_{32}$ refers to the signed, 32-bit integer type, and $I_{64}$ refers to signed, 64-bit 
integer type.
The parameters, $u_i$ or $A_{ij}$, are the stored values from the containers 
where the $i$ and $j$ parameters are set to the row and column indices 
corresponding to the location of the stored value. When operating on vectors, 
$j$ will be passed with a zero value. Finally, $s$ is an additional scalar 
value used in the operators.
The expressions in the ``Description'' column are to be treated as mathematical specifications.
    That is, for the index arithmetic functions in the first two groups below, each one of $i$, $j$, and $s$ is interpreted as an integer number in the set $\mathbb{Z}$.
    Functions are evaluated using arithmetic in $\mathbb{Z}$, producing a result value that is also in $\mathbb{Z}$. 
    The result value is converted to the output type according to the rules of the C language. In particular, if the value cannot be represented as a signed 32- or 64-bit integer type, the output is implementation defined.
    Any deviations from this ideal behavior, including limitations on the values of $i$, $j$, and $s$, or possible overflow and underflow conditions, must be defined by the implementation.
    }
\label{Tab:PredefIndexOperators}
%\vspace{1\baselineskip}

    {\small
\begin{tabular}{l|l|cccc|rcll}
Operator type             & GraphBLAS                		& \multicolumn{4}{c|}{Domains ($-$ is don't care)}	& \multicolumn{4}{c}{Description} \\ 
Type                      & Name                     		& $A,u$ & $i$, $j$  	& $s$ 		& result        & &&& \\ \hline
{\sf GrB\_IndexUnaryOp}   & {\sf GrB\_ROWINDEX\_$I_{32/64}$} 	& $-$   & $I_{U64}$	& $I_{32/64}$ 	& $I_{32/64}$ 	& $f(A_{ij},i,j,s)$ & $=$ & $(i + s)$, 		& replace with its row index (+ s) \\
                          &                          		& $-$   & $I_{U64}$ 	& $I_{32/64}$ 	& $I_{32/64}$ 	& $f(u_{i}, i,0,s)$ & $=$ & $(i + s)$  		& \\
{\sf GrB\_IndexUnaryOp}   & {\sf GrB\_COLINDEX\_$I_{32/64}$} 	& $-$   & $I_{U64}$ 	& $I_{32/64}$ 	& $I_{32/64}$ 	& $f(A_{ij},i,j,s)$ & $=$ & $(j + s)$ 		& replace with its column index (+ s) \\
{\sf GrB\_IndexUnaryOp}   & {\sf GrB\_DIAGINDEX\_$I_{32/64}$}	& $-$   & $I_{U64}$ 	& $I_{32/64}$ 	& $I_{32/64}$ 	& $f(A_{ij},i,j,s)$ & $=$ & $(j - i + s)$	& replace with its diagonal index (+ s) \\
\hline

{\sf GrB\_IndexUnaryOp}   & {\sf GrB\_TRIL}    			& $-$ 	& $I_{U64}$ 	& $I_{64}$ 	& {\sf bool} 	& $f(A_{ij},i,j,s)$ & $=$ & $(j \leq i + s)$ 	& triangle on or below diagonal s \\
{\sf GrB\_IndexUnaryOp}   & {\sf GrB\_TRIU}    			& $-$ 	& $I_{U64}$ 	& $I_{64}$ 	& {\sf bool} 	& $f(A_{ij},i,j,s)$ & $=$ & $(j \geq i + s)$ 	& triangle on or above diagonal s \\
{\sf GrB\_IndexUnaryOp}   & {\sf GrB\_DIAG}    			& $-$ 	& $I_{U64}$ 	& $I_{64}$ 	& {\sf bool} 	& $f(A_{ij},i,j,s)$ & $=$ & $(j  ==  i + s)$ 	& diagonal s \\
{\sf GrB\_IndexUnaryOp}   & {\sf GrB\_OFFDIAG} 			& $-$ 	& $I_{U64}$ 	& $I_{64}$ 	& {\sf bool} 	& $f(A_{ij},i,j,s)$ & $=$ & $(j \neq i + s)$ 	& all but diagonal s \\

{\sf GrB\_IndexUnaryOp}   & {\sf GrB\_COLLE}   			& $-$ 	& $I_{U64}$ 	& $I_{64}$ 	& {\sf bool} 	& $f(A_{ij},i,j,s)$ & $=$ & $(j \leq s)$ 	& columns less or equal to s \\
{\sf GrB\_IndexUnaryOp}   & {\sf GrB\_COLGT}   			& $-$ 	& $I_{U64}$ 	& $I_{64}$ 	& {\sf bool} 	& $f(A_{ij},i,j,s)$ & $=$ & $(j >    s)$ 	& columns greater than s \\
{\sf GrB\_IndexUnaryOp}   & {\sf GrB\_ROWLE}   			& $-$ 	& $I_{U64}$ 	& $I_{64}$ 	& {\sf bool} 	& $f(A_{ij},i,j,s)$ & $=$ & $(i \leq s)$, 	& rows less or equal to s \\
                          &                    			& $-$ 	& $I_{U64}$ 	& $I_{64}$ 	& {\sf bool} 	& $f(u_{i}, i,0,s)$ & $=$ & $(i \leq s)$  \\
{\sf GrB\_IndexUnaryOp}   & {\sf GrB\_ROWGT}   			& $-$ 	& $I_{U64}$ 	& $I_{64}$ 	& {\sf bool} 	& $f(A_{ij},i,j,s)$ & $=$ & $(i >    s)$, 	& rows greater than s \\
                          &                    			& $-$ 	& $I_{U64}$ 	& $I_{64}$ 	& {\sf bool} 	& $f(u_{i}, i,0,s)$ & $=$ & $(i >    s)$ \\
\hline
                     
{\sf GrB\_IndexUnaryOp}   & {\sf GrB\_VALUEEQ\_$T$} 		& $T$ 	& $-$ 		& $T$	 	& {\sf bool} 	& $f(A_{ij},i,j,s)$ & $=$ & $(A_{ij} ==   s)$, 	& elements equal to value s \\
                          &                         		& $T$ 	& $-$ 		& $T$ 		& {\sf bool} 	& $f(u_{i}, i,0,s)$ & $=$ & $(u_{i}  ==   s)$ \\
{\sf GrB\_IndexUnaryOp}   & {\sf GrB\_VALUENE\_$T$} 		& $T$ 	& $-$ 		& $T$ 		& {\sf bool} 	& $f(A_{ij},i,j,s)$ & $=$ & $(A_{ij} \neq s)$, 	& elements not equal to value s \\
                          &                         		& $T$ 	& $-$ 		& $T$ 		& {\sf bool} 	& $f(u_{i}, i,0,s)$ & $=$ & $(u_{i}  \neq s)$ \\
{\sf GrB\_IndexUnaryOp}   & {\sf GrB\_VALUELT\_$T$} 		& $T$ 	& $-$ 		& $T$ 		& {\sf bool} 	& $f(A_{ij},i,j,s)$ & $=$ & $(A_{ij} <    s)$, 	& elements less than value s \\
                          &                         		& $T$ 	& $-$ 		& $T$ 		& {\sf bool} 	& $f(u_{i}, i,0,s)$ & $=$ & $(u_{i}  <    s)$ \\
{\sf GrB\_IndexUnaryOp}   & {\sf GrB\_VALUELE\_$T$} 		& $T$ 	& $-$ 		& $T$ 		& {\sf bool} 	& $f(A_{ij},i,j,s)$ & $=$ & $(A_{ij} \leq s)$, 	& elements less or equal to value s \\
                          &                         		& $T$ 	& $-$ 		& $T$ 		& {\sf bool} 	& $f(u_{i}, i,0,s)$ & $=$ & $(u_{i}  \leq s)$ \\
{\sf GrB\_IndexUnaryOp}   & {\sf GrB\_VALUEGT\_$T$} 		& $T$ 	& $-$ 		& $T$ 		& {\sf bool}	& $f(A_{ij},i,j,s)$ & $=$ & $(A_{ij} >    s)$,	& elements greater than value s \\
                          &                         		& $T$ 	& $-$	 	& $T$ 		& {\sf bool} 	& $f(u_{i}, i,0,s)$ & $=$ & $(u_{i}  >    s)$ \\
{\sf GrB\_IndexUnaryOp}   & {\sf GrB\_VALUEGE\_$T$} 		& $T$ 	& $-$ 		& $T$ 		& {\sf bool} 	& $f(A_{ij},i,j,s)$ & $=$ & $(A_{ij} \geq s)$, 	& elements greater or equal to value s \\
                          &                         		& $T$ 	& $-$ 		& $T$ 		& {\sf bool} 	& $f(u_{i}, i,0,s)$ & $=$ & $(u_{i}  \geq s)$ \\
\end{tabular}
    }
\hrule
\end{threeparttable}
\end{table}


\end{landscape}

%-----------------------------------------------------------------------------
%----------------------------------------------------------------------------
\subsection{Monoids}

A GraphBLAS \emph{monoid} $M =
\langle D,\odot,0 \rangle$ is defined by a single domain $D$, an 
\emph{associative}\footnote{\label{Foot:associative}It is expected 
that implementations of the GraphBLAS will utilize floating point arithmetic 
such as that defined in the IEEE-754 standard even though
floating point arithmetic is not strictly associative.} 
operation $\odot: D \times D \rightarrow D$,
and an identity element $0 \in D$.  For a given GraphBLAS monoid $M=\langle
D,\odot,0 \rangle$ we define $\mathbf{D}(M) = D$, $\mathbf{\bigodot}(M) =
\odot$, and $\mathbf{0}(M) = 0$.  A GraphBLAS monoid is equivalent to 
the conventional \emph{monoid} algebraic structure.

Let $F = \langle D,D,D,\odot \rangle$ be an associative GraphBLAS binary operator
with identity element $0 \in D$.  Then $M = \langle F,0 \rangle = \langle
D,\odot,0 \rangle$ is a GraphBLAS monoid. If $\odot$ is commutative,
then $M$ is said to be a \emph{commutative monoid}.
If a monoid $M$ is created using an operator $\odot$ that is
not associative, the outcome of GraphBLAS operations using such a monoid is undefined.

User-defined monoids can be created with calls to {\sf GrB\_Monoid\_new} 
(see Section~\ref{Sec:AlgebraMethods}).
The GraphBLAS C API predefines a number of monoids that are listed 
in Table~\ref{Tab:PredefinedMonoids}.  Predefined monoids are named {\sf
GrB\_\emph{op}\_MONOID\_$T$}, where \emph{op} is the name of the
predefined GraphBLAS operator used as the associative binary operation
of the monoid and $T$ is the domain (type) of the monoid.

%==================

\begin{table}
\centering
\begin{threeparttable}
\hrule
\caption[Predefined monoids for GraphBLAS in C.]{Predefined monoids for GraphBLAS in C. Maximum and minimum values for the 
various integral types are defined in {\tt stdint.h}. Floating-point infinities are 
defined in {\tt math.h}. The $x$ in {\sf UINT}$x$ or {\sf INT}$x$ can be one of 8, 
16, 32, or 64; whereas in {\sf FP}$x$, it can be 32 or 64.}
\label{Tab:PredefinedMonoids}
\vspace{1\baselineskip}

\begin{tabular}{l|l|l|l}
GraphBLAS                   & Domains, $T$           &               & \\
identifier                  & ($T \times T \rightarrow T$) & Identity      & Description \\ \hline
{\sf GrB\_PLUS\_MONOID\_$T$}  & {\sf UINT}$x$  & 0    & addition \\
                            & {\sf INT}$x$   & 0    & \\
                            & {\sf FP}$x$    & 0    & \\
{\sf GrB\_TIMES\_MONOID\_$T$} & {\sf UINT}$x$  & 1    & multiplication \\
                            & {\sf INT}$x$   & 1    & \\
                            & {\sf FP}$x$    & 1    & \\
{\sf GrB\_MIN\_MONOID\_$T$}   & {\sf UINT}$x$  & {\tt UINT$x$\_MAX}  & minimum \\
                            & {\sf INT}$x$   & {\tt INT$x$\_MAX}  & \\
                            & {\sf FP}$x$    & {\tt INFINITY}   & \\
{\sf GrB\_MAX\_MONOID\_$T$}   & {\sf UINT}$x$  & 0                & maximum \\
                            & {\sf INT}$x$   & {\tt INT$x$\_MIN}  & \\
                            & {\sf FP}$x$    & {\tt -INFINITY}   & \\
\comment{
{\sf GrB\_ANY\_MONOID\_$T$}   & $T$    & (implicit)   & either input\tnote{1} \\
                            & & & \\
}
                               & & & \\
{\sf GrB\_LOR\_MONOID\_BOOL}   & {\sf BOOL}  & {\tt false}   & logical OR \\
{\sf GrB\_LAND\_MONOID\_BOOL}  & {\sf BOOL}  & {\tt true}    & logical AND \\
{\sf GrB\_LXOR\_MONOID\_BOOL}  & {\sf BOOL}  & {\tt false}   & logical XOR (not equal) \\
{\sf GrB\_LXNOR\_MONOID\_BOOL} & {\sf BOOL}  & {\tt true}    & logical XNOR (equal) \\
\end{tabular}
\hrule
\comment{
\begin{tablenotes}
    \item[1] For {\sf GrB\_ANY\_MONOID\_T}, an implementation is free to return either input operand, and is not required to always return the same operand in different invocations.  The identity of this monoid is not defined and therefore should not be used in {\sf GrB\_reduce} variants that produce scalars (where an identity could be needed).
\end{tablenotes}
}
\end{threeparttable}
\end{table}

%----------------------------------------------------------------------------
\subsection{Semirings}

A GraphBLAS \emph{semiring}
$S=\langle \Dout, \Din1, \Din2, \oplus, \otimes, 0 \rangle$ is defined by
three domains $\Dout$, $\Din1$, and $\Din2$; an \emph{associative}\footnotemark[\value{footnote}]
and commutative
additive operation $\oplus : \Dout \times \Dout \rightarrow \Dout$; 
a multiplicative operation $\otimes : \Din1 \times \Din2 \rightarrow
\Dout$; and an identity element $0 \in \Dout$.
For a given GraphBLAS semiring $S=\langle \Dout, \Din1,
\Din2, \oplus,\otimes,0 \rangle$ we define $\bDin1(S) = \Din1$,
$\bDin2(S) = \Din2$, $\bDout(S) = \Dout$, $\mathbf{\bigoplus}(S) =
\oplus$, $\mathbf{\bigotimes}(S) = \otimes$, and $\zero(S) = 0$. 

Let $F = \langle \Dout,\Din1,\Din2,\otimes \rangle$ be an operator
and let $A = \langle \Dout,\oplus,0 \rangle$ be a commutative monoid,
then $S= \langle A,F \rangle = \langle \Dout,\Din1,\Din2,\oplus,\otimes,0 \rangle$
is a semiring.

In a GraphBLAS semiring, the multiplicative operator does not have to distribute over the additive operator. 
This is unlike the conventional \emph{semiring} algebraic structure.

Note: There must be one GraphBLAS monoid in every semiring which 
serves as the semiring's additive operator and  
specifies the same domain for its inputs and output parameters. 
If this monoid is not a commutative monoid, the outcome of GraphBLAS
operations using the semiring is undefined.

%A UML diagram of the conceptual hierarchy of object classes in GraphBLAS
%algebra (binary operators, monoids, and semirings) is shown in 
%Figure~\ref{Fig:AlgebraHierarchy}.

%\begin{figure}[htb]
%    \hrule
%    \begin{center}
%        \includegraphics[width=1.0\linewidth,trim=3in 2in 0.5in 2in]{Algebra_Hierarchy_v2_1.pdf}
%    \end{center}
%    \caption[Hierarchy of algebraic object classes in GraphBLAS.]{Hierarchy of algebraic object classes in GraphBLAS. GraphBLAS 
%    semirings consist of a conventional monoid with one domain for the addition 
%    function, and a binary operator with three domains for the multiplication function.}
%    \label{Fig:AlgebraHierarchy}
%    \hrule
%\end{figure}

User-defined semirings can be created with calls to {\sf GrB\_Semiring\_new} 
(see Section~\ref{Sec:AlgebraMethods}).
A list of predefined true semirings and convenience
semirings can be found in Tables~\ref{Tab:PredefinedTrueSemirings} and~\ref{Tab:PredefinedUsefulSemirings},
respectively.  Predefined
semirings are named {\sf GrB\_\emph{add}\_\emph{mul}\_SEMIRING\_$T$},
where \emph{add} is the semiring additive operation, \emph{mul} is
the semiring multiplicative operation and $T$ is the domain (type)
of the semiring.

%==================

\begin{table}
\centering
\begin{threeparttable}
\hrule
\caption[Predefined ``true'' semirings for GraphBLAS in C.]{Predefined true semirings 
for GraphBLAS in C where the additive identity is the multiplicative 
annihilator. The $x$ can be one of 8, 16, 32, or 64 in {\sf UINT$x$} or {\sf INT$x$}, 
and can be 32 or 64 in {\sf FP$x$}.}
\label{Tab:PredefinedTrueSemirings}

\hspace*{-1.5em}
\begin{tabular}{l|l|l|l}
                                      & Domains, $T$             & $+$ identity         &                 \\
GraphBLAS identifier              & ($T \times T \rightarrow T$) & $\times$ annihilator & Description     \\ \hline
{\sf GrB\_PLUS\_TIMES\_SEMIRING\_$T$}   & {\sf UINT$x$}            & 0                    & arithmetic semiring \\
                                      & {\sf INT$x$}             & 0                    &                 \\
                                      & {\sf FP$x$}              & 0                    &                 \\
{\sf GrB\_MIN\_PLUS\_SEMIRING\_$T$}     & {\sf UINT$x$}            & {\tt UINT$x$\_MAX}   & min-plus semiring  \\
                                      & {\sf INT$x$}             & {\tt INT$x$\_MAX}    &                 \\
                                      & {\sf FP$x$}              & {\tt INFINITY}       &                 \\
{\sf GrB\_MAX\_PLUS\_SEMIRING\_$T$}     & {\sf INT$x$}             & {\tt INT$x$\_MIN}    & max-plus semiring  \\
                                      & {\sf FP$x$}              & {\tt -INFINITY}      &                 \\
{\sf GrB\_MIN\_TIMES\_SEMIRING\_$T$}    & {\sf UINT$x$}            & {\tt UINT$x$\_MAX}   & min-times semiring \\
{\sf GrB\_MIN\_MAX\_SEMIRING\_$T$}      & {\sf UINT$x$}            & {\tt UINT$x$\_MAX}   & min-max semiring   \\
                                      & {\sf INT$x$}             & {\tt INT$x$\_MAX}    &                 \\
                                      & {\sf FP$x$}              & {\tt INFINITY}       &                 \\
{\sf GrB\_MAX\_MIN\_SEMIRING\_$T$}      & {\sf UINT$x$}            & 0                    & max-min semiring   \\
                                      & {\sf INT$x$}             & {\tt INT$x$\_MIN}    &                 \\
                                      & {\sf FP$x$}              & {\tt -INFINITY}      &                 \\
{\sf GrB\_MAX\_TIMES\_SEMIRING\_$T$}    & {\sf UINT$x$}            & 0                    & max-times semiring \\
{\sf GrB\_PLUS\_MIN\_SEMIRING\_$T$}     & {\sf UINT$x$}            & 0                    & plus-min semiring  \\
                                      &                          &                      &                 \\
{\sf GrB\_LOR\_LAND\_SEMIRING\_BOOL}  & {\sf BOOL}               & {\tt false}          & Logical semiring   \\
{\sf GrB\_LAND\_LOR\_SEMIRING\_BOOL}  & {\sf BOOL}               & {\tt true}           & "and-or" semiring  \\
{\sf GrB\_LXOR\_LAND\_SEMIRING\_BOOL} & {\sf BOOL}               & {\tt false}          & same as {\sf NE\_LAND} \\
{\sf GrB\_LXNOR\_LOR\_SEMIRING\_BOOL} & {\sf BOOL}               & {\tt true}           & same as {\sf EQ\_LOR} \\
\end{tabular}

\hrule
\comment{
\begin{tablenotes}
    \item[1] For {\sf GrB\_ANY\_*\_SEMIRING\_T}, an implementation is free to return any of the results of the application of the "multiply" operator ({\sf FIRST} or {\sf SECOND}), and is not required to always return the same result in different invocations..
\end{tablenotes}
}
\end{threeparttable}
\end{table}

\begin{table}
\centering
\begin{threeparttable}
\hrule
\caption[Other useful predefined semirings for GraphBLAS in C.]{Other useful predefined semirings for GraphBLAS in C that don't have a multiplicative annihilator. 
The $x$ can be one of 8, 16, 32, or 64 in {\sf UINT$x$} or {\sf INT$x$}, 
and can be 32 or 64 in {\sf FP$x$}.}
\label{Tab:PredefinedUsefulSemirings}

\hspace*{-1.5em}
\begin{tabular}{l|l|l|l}
                                    & Domains, $T$             &            &                 \\
GraphBLAS identifier           & ($T \times T \rightarrow T$)  & $+$ identity      & Description             \\ \hline
{\sf GrB\_MAX\_PLUS\_SEMIRING\_$T$}   & {\sf UINT$x$}            & 0                 & max-plus semiring         \\
{\sf GrB\_MIN\_TIMES\_SEMIRING\_$T$}  & {\sf INT$x$}             & {\tt INT$x$\_MAX} & min-times semiring        \\
                                    & {\sf FP$x$}              & {\tt INFINITY}    &                  \\
{\sf GrB\_MAX\_TIMES\_SEMIRING\_$T$}  & {\sf INT$x$}             & {\tt INT$x$\_MIN} & max-times semiring        \\
                                    & {\sf FP$x$}              & {\tt -INFINITY}   &                 \\
{\sf GrB\_PLUS\_MIN\_SEMIRING\_$T$}   & {\sf INT$x$}             & 0                 & plus-min semiring          \\
                                    & {\sf FP$x$}              & 0                 &                 \\ 
{\sf GrB\_MIN\_FIRST\_SEMIRING\_$T$}  & {\sf UINT$x$}            & {\tt UINT$x$\_MAX}& min-select first  semiring     \\
                                    & {\sf INT$x$}             & {\tt INT$x$\_MAX} &                 \\
                                    & {\sf FP$x$}              & {\tt INFINITY}    &                 \\
{\sf GrB\_MIN\_SECOND\_SEMIRING\_$T$} & {\sf UINT$x$}            & {\tt UINT$x$\_MAX}& min-select second semiring     \\
                                    & {\sf INT$x$}             & {\tt INT$x$\_MAX} &                 \\
                                    & {\sf FP$x$}              & {\tt INFINITY}    &                 \\
{\sf GrB\_MAX\_FIRST\_SEMIRING\_$T$}  & {\sf UINT$x$}            & 0                 & max-select first  semiring     \\
                                    & {\sf INT$x$}             & {\tt INT$x$\_MIN} &                 \\
                                    & {\sf FP$x$}              & {\tt -INFINITY}   &                 \\
{\sf GrB\_MAX\_SECOND\_SEMIRING\_$T$} & {\sf UINT$x$}            & 0                 & max-select second semiring     \\
                                    & {\sf INT$x$}             & {\tt INT$x$\_MIN} &                 \\
                                    & {\sf FP$x$}              & {\tt -INFINITY}   &                 \\
\end{tabular}

\hrule
\comment{
\begin{tablenotes}
    \item[1] For {\sf GrB\_ANY\_*\_SEMIRING\_T}, an implementation is free to return any of the results of the application of the "multiply" operator ({\sf FIRST} or {\sf SECOND}), and is not required to always return the same result in different invocations..
\end{tablenotes}
}
\end{threeparttable}
\end{table}

%============================================================================
\section{Collections}

%----------------------------------------------------------------------------
\subsection{Scalars}
\label{Sec:Scalars}

A \emph{GraphBLAS scalar}, $\scalar{s} = \langle D, \{ \sigma \} \rangle$, is defined by
a domain $D$, and a set of zero or one \emph{scalar value}, $\sigma$, where $\sigma \in D$. 
We define $\mathbf{size}(\scalar{s}) = 1$ (constant), and
$\mathbf{L}(\scalar{s}) = \{ \sigma \}$. The set $\mathbf{L}(\scalar{s})$ is
called the \emph{contents} of the GraphBLAS scalar $\scalar{s}$. We also define 
$\mathbf{D}(\scalar{s}) = D$. Finally, $\mathbf{val}(s)$ is a 
reference to the scalar value, $\sigma$, if the GraphBLAS scalar is not empty, and is 
undefined otherwise.

%----------------------------------------------------------------------------
\subsection{Vectors}
\label{Sec:Vectors}

A vector $\vector{v} = \langle D, N, \{ (i,v_i) \} \rangle$ is defined by
a domain $D$, a size $N>0$, and a set of tuples $(i,v_i)$ where $0 \leq
i < N$ and $v_i \in D$. A particular value of $i$ can appear at
most once in $\vector{v}$. We define $\mathbf{size}(\vector{v}) = N$ and
$\mathbf{L}(\vector{v}) = \{ (i,v_i) \}$. The set $\mathbf{L}(\vector{v})$ is
called the \emph{content} of vector $\vector{v}$. We also define the set
$\vector{ind(\vector{v})} = \{ i : (i,v_i) \in \mathbf{L}(\vector{v}) \}$
(called the \emph{structure} of $\vector{v}$), and $\mathbf{D}(\vector{v})
= D$. For a vector $\vector{v}$, $\vector{v}(i)$ is a reference to $v_i$
if $(i,v_i) \in \mathbf{L}(\vector{v})$ and is undefined otherwise.

%----------------------------------------------------------------------------
\subsection{Matrices}
\label{Sec:Matrices}

A matrix $\matrix{A} = \langle D, M, N, \{ (i,j,A_{ij}) \} \rangle$ is
defined by a domain $D$, its number of rows $M>0$, its number of columns
$N>0$, and a set of tuples $(i,j,A_{ij})$ where $0 \leq i < M$, $0 \leq
j < N$, and $A_{ij} \in D$. A particular pair of values $i,j$ can
appear at most once in $\matrix{A}$. We define $\mathbf{ncols}(\matrix{A})
= N$,  $\mathbf{nrows}(\matrix{A}) = M$, and $\mathbf{L}(\matrix{A}) =
\{ (i,j,A_{ij}) \}$.  The set $\mathbf{L}(\matrix{A})$ is called the
\emph{content} of matrix $\matrix{A}$.  We also define the sets
$\vector{indrow(\matrix{A})} = \{ i : \exists (i,j,A_{ij}) \in
\matrix{A} \}$ and $\vector{indcol(\matrix{A})} = \{ j : \exists
(i,j,A_{ij}) \in \matrix{A} \}$.  (These are the sets of nonempty
rows and columns of $\matrix{A}$, respectively.)  The \emph{structure}
of matrix $\matrix{A}$ is the set $\mathbf{ind}(\matrix{A}) = \{ (i,j) :
(i,j,A_{ij}) \in \mathbf{L}(\matrix{A}) \}$, and $\mathbf{D}(\matrix{A}) = D$.
For a matrix $\matrix{A}$, $\matrix{A}(i,j)$ is a reference to $A_{ij}$
if $(i,j,A_{ij}) \in \mathbf{L}(\matrix{A})$ and is undefined otherwise.

If $\matrix{A}$ is a matrix and $0 \leq j < N$, then $\matrix{A}(:,j)
= \langle D, M, \{(i,A_{ij}) : (i,j,A_{ij}) \in \mathbf{L}(\matrix{A})
\} \rangle$ is a vector called the $j$-th \emph{column}
of $\matrix{A}$. Correspondingly, if $\matrix{A}$ is a matrix and
$0 \leq i < M$, then $\matrix{A}(i,:) = \langle D, N, \{(j,A_{ij}) :
(i,j,A_{ij}) \in \mathbf{L}(\matrix{A}) \} \rangle$ is a vector called
the $i$-th \emph{row} of $\matrix{A}$.

Given a matrix $\matrix{A} = \langle D, M, N, \{ (i,j,A_{ij}) \} \rangle$,
its \emph{transpose} is another matrix $\matrix{A}^T = \langle D, N, M, \{
(j,i,A_{ij}) : (i,j,A_{ij}) \in \mathbf{L}(\matrix{A}) \} \rangle$.

%----------------------------------------------------------------------------
\subsubsection{External matrix formats}\label{Sec:GrB_Format}

The specification also supports the export and import of matrices to/from a 
number of commonly used formats, such as COO, CSR, and CSC formats.  When
importing or exporting a matrix to or from a GraphBLAS object using
{\sf GrB\_Matrix\_import} (\S~\ref{Sec:Matrix_import}) or {\sf GrB\_Matrix\_export} (\S~\ref{Sec:Matrix_export}), it is necessary to
specify the data format for the matrix data external to GraphBLAS, which is
being imported from or exported to.  This non-opaque data format is specified
using an argument of enumeration type {\sf GrB\_Format} that is used to 
indicate one of a number of predefined formats.  The predefined values of 
{\sf GrB\_Format} are specified in Table~\ref{Tab:MatrixFormatEnumerationValues}.  
A precise definition of the non-opaque data formats can be found in 
Appendix~\ref{App:Matrix_format_details}.

\begin{table}[bh]
\hrule
\begin{center}
\caption{{\sf GrB\_Format} enumeration literals and corresponding values for 
matrix import and export methods.}
\label{Tab:MatrixFormatEnumerationValues}

\begin{tabular}{l|r|p{3.75in}}
Symbol    & Value & Description \\ \hline
{\sf GrB\_CSR\_FORMAT} & 0 & Specifies the compressed sparse row matrix format.\\
{\sf GrB\_CSC\_FORMAT} & 1 & Specifies the compressed sparse column matrix format.\\
{\sf GrB\_COO\_FORMAT} & 2 & Specifies the sparse coordinate matrix format.\\
% {\sf GrB\_DENSE\_ROW\_FORMAT} & 3 & Specifies the dense row-major matrix format.\\
% {\sf GrB\_DENSE\_COL\_FORMAT} & 4 & Specifies the dense column-major matrix format.\\
\end{tabular}

\end{center}
\hrule
\end{table}


%----------------------------------------------------------------------------
\subsection{Masks}
\label{Sec:Masks}

The GraphBLAS C API defines an opaque object called a \emph{mask}.  The mask
is used to control how computed values are stored in the output from a method. 
The mask is an \emph{internal} opaque object; that is, it is never exposed as a 
variable within an application. 

The mask is formed from input objects to the method that uses 
the mask.  For example, a GraphBLAS method may be called with a matrix as the mask
parameter.   The internal mask object is constructed from the input matrix in one
of two ways.  In the default case, an element of the mask is created for each 
tuple that exists in the matrix for which the value of the tuple cast to Boolean 
evaluates to {\tt true}.  Alternatively, the user can specify {\em structure}-only 
behavior where an element of the mask is created for each tuple that exists in 
the matrix {\em regardless} of the value stored in the input matrix.

The internal mask object can be either a one- or a two-dimensional construct.  
One- and two-dimensional masks, described more formally below, are similar to
vectors and matrices, respectively, except that they have structure
(indices) but no values.  When needed, a value is implied for the elements of a 
mask with an implied value of {\tt true} for elements that exist 
and an implied value of {\tt false} for elements that do not exist (\ie,
the locations of the mask that do not have a stored value imply a value of {\tt false}).
Hence, even though a mask does not contain any values, it can be 
considered to imply values from a Boolean domain.

A one-dimensional mask $\vector{m} = \langle N, \{ i \} \rangle$ is
defined by its number of elements $N>0$, and a set $\mathbf{ind}(\vector{m})$
of indices $\{ i \}$ where $0 \leq i < N$.  A particular value of $i$ can
appear at most once in $\vector{m}$. We define $\mathbf{size}(\vector{m})
= N$. The set $\mathbf{ind}(\vector{m})$ is called the \emph{structure} of mask $\vector{m}$.

A two-dimensional mask $\matrix{M} = \langle M, N, \{ (i,j) \}
\rangle$ is defined by its number of rows $M>0$, its number of
columns $N>0$, and a set $\mathbf{ind}(\matrix{M})$ of tuples $(i,j)$
where $0 \leq i < M$, $0 \leq j < N$.   A particular pair of values
$i,j$ can appear at most once in $\matrix{M}$.  We define
$\mathbf{ncols}(\matrix{M}) = N$, and $\mathbf{nrows}(\matrix{M}) = M$.
We also define the sets $\vector{indrow(\matrix{M})} = \{ i : \exists
(i,j) \in \mathbf{ind}(\matrix{M}) \}$ and $\vector{indcol(\matrix{M})}
= \{ j : \exists (i,j) \in \mathbf{ind}(\matrix{M}) \}$.  These are
the sets of nonempty rows and columns of $\matrix{M}$, respectively.
The set $\mathbf{ind}(\matrix{M})$ is called the \emph{structure} of 
mask $\matrix{M}$.

One common operation on masks is the \emph{complement}.
For a one-dimensional mask $\vector{m}$ this is denoted as
$\neg\vector{m}$. For a two-dimensional mask $\matrix{M}$, this is denoted as
$\neg\matrix{M}$.  The complement of a one-dimensional
mask $\vector{m}$ is defined as $\mathbf{ind}(\neg\vector{m}) = \{i : 0
\leq i < N, i \notin \mathbf{ind}(\vector{m}) \}$.  It is the set of all
possible indices that do not appear in $\vector{m}$.  The 
complement of a two-dimensional mask $\matrix{M}$ is defined as the set
$\mathbf{ind}(\neg\matrix{M}) = \{(i,j)$ : $0 \leq i < M$, $0 \leq j < N$,
$(i,j) \notin \mathbf{ind}(\matrix{M}) \}$.  It is the set of all possible
indices that do not appear in $\matrix{M}$.

%----------------------------------------------------------------------------
%============================================================================
\section{Descriptors}
\label{Sec:Descriptors}

Descriptors are used to modify the behavior of a GraphBLAS method.
When present in the signature of a method, they appear as the last
argument in the method.  Descriptors specify how the other input arguments
corresponding to GraphBLAS collections -- vectors, matrices, and masks
-- should be processed (modified) before the main operation of a method
is performed.  A complete list of
what descriptors are capable of are presented in this section.

The descriptor is a lightweight object.  It is composed of (\emph{field},
\emph{value}) pairs where the \emph{field} selects one of the GraphBLAS objects
from the argument list of a method and the \emph{value} defines the
indicated modification associated with that object.  For example,
a descriptor may specify that a particular input matrix needs to be
transposed or that a mask needs to be complemented (defined
in Section~\ref{Sec:Masks}) before using it in the operation.

For the purpose of constructing descriptors, the arguments of a method
that can be modified are identified by specific field names. The output
parameter (typically the first parameter in a GraphBLAS method) is
indicated by the field name, {\sf GrB\_OUTP}.  The mask is indicated
by the {\sf GrB\_MASK} field name. The input parameters corresponding
to the input vectors and matrices are indicated by {\sf GrB\_INP0}
and {\sf GrB\_INP1} in the order they appear in the signature of the
GraphBLAS method.  The descriptor is an opaque object and hence we do not
define how objects of this type should be implemented.   When referring to
(\emph{field}, \emph{value}) pairs for a descriptor, however, we often use the informal
notation {\sf desc[GrB\_Desc\_Field].GrB\_Desc\_Value} without implying
that a descriptor is to be implemented as an array of structures (in fact,
field values can be used in conjunction with multiple values that are composable).
We summarize all types, field names, and values used with descriptors
in Table~\ref{Tab:DescTypeLiterals}.

\begin{table}
\hrule
\begin{center}
\caption[Descriptor types and literals for fields and values.]{Descriptors are GraphBLAS objects passed as arguments to GraphBLAS 
operations to modify other GraphBLAS objects in the operation's argument list.
A descriptor, {\sf desc}, has one or more (\emph{field}, \emph{value}) pairs indicated 
as  {\sf desc[GrB\_Desc\_Field].GrB\_Desc\_Value}. In this table, we define all types and literals used
with descriptors.}
\label{Tab:DescTypeLiterals}

\vspace{1\baselineskip}
(a) Types used with GraphBLAS descriptors.
\vspace{1\baselineskip}

\begin{tabular}{l|l}
Type                      & Description \\ \hline
{\sf GrB\_Descriptor}     &  Type of a GraphBLAS descriptor object. \\
{\sf GrB\_Desc\_Field}    &  The descriptor field enumeration. \\
{\sf GrB\_Desc\_Value}    &  The descriptor value enumeration. \\
\end{tabular}

\vspace{1\baselineskip}
(b) Descriptor field names of type {\sf GrB\_Desc\_Field} enumeration and corresponding values.
\vspace{1\baselineskip}

\begin{tabular}{l|l|l}
Field Name      & Value  & Description \\ \hline
{\sf GrB\_OUTP} & 0      &  Field name for the output GraphBLAS object. \\
{\sf GrB\_MASK} & 1      &  Field name for the mask GraphBLAS object. \\
{\sf GrB\_INP0} & 2      &  Field name for the first input GraphBLAS object. \\
{\sf GrB\_INP1} & 3      &  Field name for the second input  GraphBLAS object. \\
\end{tabular}

\vspace{1\baselineskip}
(c) Descriptor field values of type {\sf GrB\_Desc\_Value} enumeration and corresponding values.
\vspace{1\baselineskip}

\begin{tabular}{l|l|l}
Value Name           & Value  & Description \\ \hline
(reserved)           & 0      & Unused \\ 
{\sf GrB\_REPLACE}   & 1      & Clear the output object before assigning computed values.\\
{\sf GrB\_COMP}      & 2      & Use the complement of the associated object. When combined \\ 
                     &        & with {\sf GrB\_STRUCTURE}, the complement of the structure of the \\
                     &        & associated object is used without evaluating the values stored. \\
{\sf GrB\_TRAN}      & 3      &  Use the transpose of the associated object.\\
{\sf GrB\_STRUCTURE} & 4      & The write mask is constructed from the structure (pattern of \\
                     &        & stored values) of the associated object. The stored values are \\
                     &        & not examined.\\
\end{tabular}
\end{center}
\hrule
\end{table}

In the definitions of the GraphBLAS methods, we often refer to the
\emph{default behavior} of a method with respect to the action of a
descriptor.   If a descriptor is not provided or if the value associated
with a particular field in a descriptor is not set, the default behavior
of a GraphBLAS method is defined as follows:
\begin{itemize}
\item Input matrices are not transposed.
\item The mask is used, as is, without complementing, and stored values are examined to 
determine whether they evaluate to {\tt true} or {\tt false}.

\item Values of the output object that are not directly modified by the operation are preserved.
\end{itemize}

GraphBLAS specifies all of the valid combinations of (field, value) pairs as 
predefined descriptors. Their identifiers and the corresponding set of 
(field, value) pairs for that identfier are shown in 
Table~\ref{Tab:DefaultDescriptors}.

\newcommand{\grboutp}{{\sf GrB\_OUTP}}
\newcommand{\grbmask}{{\sf GrB\_MASK}}
\newcommand{\grbinp}[1]{{\sf GrB\_INP#1}}
\newcommand{\grbreplace}{{\sf GrB\_REPLACE}}
\newcommand{\grbstructure}{{\sf GrB\_STRUCTURE}}
\newcommand{\grbscmp}{{\sf GrB\_COMP}}

\newcommand{\grbrepl}{{\sf GrB\_REPLACE}}
\newcommand{\grbstrc}{{\sf GrB\_STRUCTURE}}
\newcommand{\grbcomp}{{\sf GrB\_COMP}}
\newcommand{\grbtran}{{\sf GrB\_TRAN}}

\begin{table}[htbp]
    \hrule
    \begin{center}
    \caption[Predefined GraphBLAS descriptors.]{Predefined GraphBLAS descriptors. The list includes
    all possible descriptors, according to the current standard.  Columns list the
    possible fields and entries list the value(s) associated with those fields for
    a given descriptor.}
    \label{Tab:DefaultDescriptors}
~\\
    \begin{small}

        \begin{tabular}{l|llll} 
        Identifier          & {\sf GrB\_OUTP} & {\sf GrB\_MASK} & {\sf GrB\_INP0} & {\sf GrB\_INP1}  \\ \hline
        {\sf GrB\_NULL}     &    --    &    --    &    --    &    --    \\
        {\sf GrB\_DESC\_T1}       &    --    &    --    &    --    & \grbtran \\
        {\sf GrB\_DESC\_T0}       &    --    &    --    & \grbtran &    --    \\
        {\sf GrB\_DESC\_T0T1}     &    --    &    --    & \grbtran & \grbtran \\
        {\sf GrB\_DESC\_C}        &    --    & \grbcomp &    --    &    --    \\
        {\sf GrB\_DESC\_S}        &    --    & \grbstrc &    --    &    --    \\
        {\sf GrB\_DESC\_CT1}      &    --    & \grbcomp &    --    & \grbtran \\
        {\sf GrB\_DESC\_ST1}      &    --    & \grbstrc &    --    & \grbtran \\
        {\sf GrB\_DESC\_CT0}      &    --    & \grbcomp & \grbtran &    --    \\
        {\sf GrB\_DESC\_ST0}      &    --    & \grbstrc & \grbtran &    --    \\
        {\sf GrB\_DESC\_CT0T1}    &    --    & \grbcomp & \grbtran & \grbtran \\
        {\sf GrB\_DESC\_ST0T1}    &    --    & \grbstrc & \grbtran & \grbtran \\
        {\sf GrB\_DESC\_SC}       &    --    & \grbstrc, \grbcomp &    --    &    --    \\
        {\sf GrB\_DESC\_SCT1}     &    --    & \grbstrc, \grbcomp &    --    & \grbtran \\
        {\sf GrB\_DESC\_SCT0}     &    --    & \grbstrc, \grbcomp & \grbtran &    --    \\
        {\sf GrB\_DESC\_SCT0T1}   &    --    & \grbstrc, \grbcomp & \grbtran & \grbtran \\
        {\sf GrB\_DESC\_R}        & \grbrepl &    --    &    --    &    --    \\
        {\sf GrB\_DESC\_RT1}      & \grbrepl &    --    &    --    & \grbtran \\
        {\sf GrB\_DESC\_RT0}      & \grbrepl &    --    & \grbtran &    --    \\
        {\sf GrB\_DESC\_RT0T1}    & \grbrepl &    --    & \grbtran & \grbtran \\
        {\sf GrB\_DESC\_RC}       & \grbrepl & \grbcomp &    --    &    --    \\
        {\sf GrB\_DESC\_RS}       & \grbrepl & \grbstrc &    --    &    --    \\
        {\sf GrB\_DESC\_RCT1}     & \grbrepl & \grbcomp &    --    & \grbtran \\
        {\sf GrB\_DESC\_RST1}     & \grbrepl & \grbstrc &    --    & \grbtran \\
        {\sf GrB\_DESC\_RCT0}     & \grbrepl & \grbcomp & \grbtran &    --    \\
        {\sf GrB\_DESC\_RST0}     & \grbrepl & \grbstrc & \grbtran &    --    \\
        {\sf GrB\_DESC\_RCT0T1}   & \grbrepl & \grbcomp & \grbtran & \grbtran \\
        {\sf GrB\_DESC\_RST0T1}   & \grbrepl & \grbstrc & \grbtran & \grbtran \\
        {\sf GrB\_DESC\_RSC}      & \grbrepl & \grbstrc, \grbcomp &    --    &    --    \\
        {\sf GrB\_DESC\_RSCT1}    & \grbrepl & \grbstrc, \grbcomp &    --    & \grbtran \\
        {\sf GrB\_DESC\_RSCT0}    & \grbrepl & \grbstrc, \grbcomp & \grbtran &    --    \\
        {\sf GrB\_DESC\_RSCT0T1}  & \grbrepl & \grbstrc, \grbcomp & \grbtran & \grbtran \\
        \end{tabular}
    \end{small}

    \end{center}
    \hrule
\end{table}

%============================================================================
\section{{GrB\_Info} return values}

All GraphBLAS methods return a {\sf GrB\_Info} enumeration value. The three
types of return codes (informational, API error, and execution error) and their
corresponding values are listed in Table~\ref{Tab:GrBInfoValues}.

\begin{table}[bh]
\hrule
\begin{center}
\caption{Enumeration literals and corresponding values returned by GraphBLAS methods and operations.}
\label{Tab:GrBInfoValues}

\vspace{1\baselineskip}
(a) Informational return values
\vspace{1\baselineskip}

\begin{tabular}{l|r|p{4.45in}}
Symbol    & Value & Description \\ \hline
{\sf GrB\_SUCCESS}     &  0 & The method/operation completed successfully (blocking mode), or encountered no API errors (non-blocking mode). \\
{\sf GrB\_NO\_VALUE}   &  1 & A location in a matrix or vector is being accessed that has no stored value at the specified location.\\
\end{tabular}

\vspace{1\baselineskip}
(b) API errors
\vspace{1\baselineskip}

\begin{tabular}{l|r|p{3.45in}}
Symbol    & Value & Description \\ \hline
{\sf GrB\_UNINITIALIZED\_OBJECT} & -1 & A GraphBLAS object is passed to a method before {\sf new} was called on it.\\
{\sf GrB\_NULL\_POINTER}         & -2 & A NULL is passed for a pointer parameter. \\
{\sf GrB\_INVALID\_VALUE}        & -3 & Miscellaneous incorrect values. \\
{\sf GrB\_INVALID\_INDEX}        & -4 & Indices passed are larger than dimensions of the matrix or vector being accessed. \\
{\sf GrB\_DOMAIN\_MISMATCH}      & -5 & A mismatch between domains of collections and operations when user-defined domains are in use.\\
{\sf GrB\_DIMENSION\_MISMATCH}~~ & -6 & Operations on matrices and vectors with incompatible dimensions. \\
{\sf GrB\_OUTPUT\_NOT\_EMPTY}    & -7 & An attempt was made to build a matrix or vector using an output object that already contains valid tuples (elements).\\
{\sf GrB\_NOT\_IMPLEMENTED}      & -8 &  An attempt was made to call a GraphBLAS method for a combination of input parameters that is not supported by a particular implementation.\\
\end{tabular}

\vspace{1\baselineskip}
(c) Execution errors
\vspace{1\baselineskip}

\begin{tabular}{l|r|p{3.4in}}
Symbol    & Value & Description \\ \hline
{\sf GrB\_PANIC}                  & -101 & Unknown internal error. \\
{\sf GrB\_OUT\_OF\_MEMORY}        & -102 & Not enough memory for operations. \\
{\sf GrB\_INSUFFICIENT\_SPACE}    & -103 & The array provided is not large enough to hold output. \\
{\sf GrB\_INVALID\_OBJECT}        & -104 & One of the opaque GraphBLAS objects (input or output) is in an invalid state caused by a previous execution error. \\
{\sf GrB\_INDEX\_OUT\_OF\_BOUNDS} & -105 & Reference to a vector or matrix element that is outside the defined dimensions of the object. \\
{\sf GrB\_EMPTY\_OBJECT}          & -106 & One of the opaque GraphBLAS objects does not have a stored value. \\
\end{tabular}

\end{center}
\hrule
\end{table}


%-----------------------------------------------------------------------------
\chapter{LAGraph API}
\label{Chp:API}

This chapter defines the behavior of all the functions in the LAGraph library.
All methods can be declared for use in programs by including the {\tt LAGraph.h} header file.


\section{{\sf LAGraph\_ConnectedComponents}}

Finds the connected components of an undirected graph.

\paragraph{\syntax}

\begin{verbatim}

        int LAGr_ConnectedComponents
        (
              GrB_Vector *component,  
              LAGraph_Graph G,    
              char *msg
         )
\end{verbatim}

\paragraph{Parameters}

\begin{itemize}[leftmargin=1.1in]
    \item[{\sf *component}]    ({\sf OUT}) An array holding identifiers to the components.
    \item[{\sf G}] ({\sf IN}) the input Graph (not modified by this function).
    \item{\sf msg} A message meaning something.

    \end{itemize}

\paragraph{Return Values}

\begin{itemize}[leftmargin=2.1in]
    \item[{\sf GrB\_SUCCESS}]         In blocking mode, the operation completed
    successfully. In non-blocking mode, this indicates that the compatibility 
    tests on dimensions and domains for the input arguments passed successfully. 
    Either way, output matrix {\sf C} is ready to be used in the next method of
    the sequence.

    \item[{\sf GrB\_PANIC}]           Unknown internal error.

    \item[{\sf GrB\_INVALID\_OBJECT}] This is returned in any execution mode 
    whenever one of the opaque GraphBLAS objects (input or output) is in an invalid 
    state caused by a previous execution error.  Call {\sf GrB\_error()} to access 
    any error messages generated by the implementation.

    \item[{\sf GrB\_OUT\_OF\_MEMORY}] Not enough memory available for the operation.

    \item[{\sf GrB\_UNINITIALIZED\_OBJECT}] One or more of the GraphBLAS objects 
    has not been initialized by a call to {\sf new} (or {\sf Matrix\_dup} for matrix
    parameters).

    \item[{\sf GrB\_DIMENSION\_MISMATCH}] Mask and/or matrix
    dimensions are incompatible.

    \item[{\sf GrB\_DOMAIN\_MISMATCH}]    The domains of the various matrices are
    incompatible with the corresponding domains of the semiring or
    accumulation operator, or the mask's domain is not compatible with {\sf bool}
    (in the case where {\sf desc[GrB\_MASK].GrB\_STRUCTURE} is not set).
\end{itemize}

\paragraph{Description}

{\sf GrB\_mxm} computes the matrix product ${\sf C} = {\sf
A} \oplus . \otimes {\sf B}$ or, if an optional binary accumulation
operator ($\odot$) is provided, ${\sf C} = {\sf C} \odot
\left({\sf A} \oplus . \otimes {\sf B}\right)$ (where matrices {\sf A}
and {\sf B} can be optionally transposed).  Logically, this operation
occurs in three steps:
\begin{enumerate}[leftmargin=0.85in]
\item[\bf Setup] The internal matrices and mask used in the computation are formed and their 
domains and dimensions are tested for compatibility.
\item[\bf Compute] The indicated computations are carried out.
\item[\bf Output] The result is written into the output matrix, possibly under control of a mask.
\end{enumerate}

Up to four argument matrices are used in the {\sf GrB\_mxm} operation:
\begin{enumerate}
	\item ${\sf C} = \langle \bold{D}({\sf C}),\bold{nrows}({\sf C}),\bold{ncols}({\sf C}),\bold{L}({\sf C}) = \{(i,j,C_{ij}) \} \rangle$
	\item ${\sf Mask} = \langle \bold{D}({\sf Mask}),\bold{nrows}({\sf Mask}),\bold{ncols}({\sf Mask}),\bold{L}({\sf Mask}) = \{(i,j,M_{ij}) \} \rangle$ (optional)
	\item ${\sf A} = \langle \bold{D}({\sf A}),\bold{nrows}({\sf A}), \bold{ncols}({\sf A}),\bold{L}({\sf A}) = \{(i,j,A_{ij}) \} \rangle$
	\item ${\sf B} = \langle \bold{D}({\sf B}),\bold{nrows}({\sf B}), \bold{ncols}({\sf B}),\bold{L}({\sf B}) = \{(i,j,B_{ij}) \} \rangle$
\end{enumerate}


From this point forward, in {\sf GrB\_NONBLOCKING} mode, the method can 
optionally exit with {\sf GrB\_SUCCESS} return code and defer any computation 
and/or execution error codes.

We are now ready to carry out the matrix multiplication and any additional 
associated operations.  We describe this in terms of two intermediate matrices:
\begin{itemize}
    \item $\matrix{\widetilde{T}}$: The matrix holding the product of matrices 
    $\matrix{\widetilde{A}}$ and $\matrix{\widetilde{B}}$.
    \item $\matrix{\widetilde{Z}}$: The matrix holding the result after 
    application of the (optional) accumulation operator.
\end{itemize}

The intermediate matrix $\matrix{\widetilde{T}} = \langle
\bDout({\sf op}), \bold{nrows}(\matrix{\widetilde{A}}), \bold{ncols}(\matrix{\widetilde{B}}),
\{(i,j,T_{ij}) : \bold{ind}(\matrix{\widetilde{A}}(i,:)) \cap
\bold{ind}(\matrix{\widetilde{B}}(:,j)) \neq \emptyset \} \rangle$
is created.  The value of each of its elements is computed by 
\[T_{ij} = \bigoplus_{k \in \bold{ind}(\matrix{\widetilde{A}}(i,:)) \cap
\bold{ind}(\matrix{\widetilde{B}}(:,j))} (\matrix{\widetilde{A}}(i,k)
\otimes \matrix{\widetilde{B}}(k,j)),\] where $\oplus$ and $\otimes$
are the additive and multiplicative operators of semiring {\sf op},
respectively.


%-----------------------------------------------------------------------------

\subsection{{\sf vxm}: Vector-matrix multiply}

Multiplies a (row) vector with a matrix on an semiring. The result is a vector.

\paragraph{\syntax}

\begin{verbatim}
        GrB_Info GrB_vxm(GrB_Vector             w,
                         const GrB_Vector       mask,
                         const GrB_BinaryOp     accum,
                         const GrB_Semiring     op,
                         const GrB_Vector       u, 
                         const GrB_Matrix       A,
                         const GrB_Descriptor   desc);
\end{verbatim}

\paragraph{Parameters}

\begin{itemize}[leftmargin=1.1in]
    \item[{\sf w}]    ({\sf INOUT}) An existing GraphBLAS vector.  On input,
    the vector provides values that may be accumulated with the result of the
    vector-matrix product.  On output, this vector holds the results of the
    operation.

    \item[{\sf mask}] ({\sf IN}) An optional ``write'' mask that controls which
    results from this operation are stored into the output vector {\sf w}. The 
    mask dimensions must match those of the vector {\sf w}. If the 
    {\sf GrB\_STRUCTURE} descriptor is {\em not} set for the mask, the domain of the
    {\sf mask} vector must be of type {\sf bool} or any of the predefined 
    ``built-in'' types in Table~\ref{Tab:PredefinedTypes}.  If the default
    mask is desired (\ie, a mask that is all {\sf true} with the dimensions of {\sf w}), 
    {\sf GrB\_NULL} should be specified.

    \item[{\sf accum}] ({\sf IN}) An optional binary operator used for accumulating
    entries into existing {\sf w} entries.
    %: ${\sf accum} = \langle \bDout({\sf accum}),\bDin1({\sf accum}),
    %\bDin2({\sf accum}), \odot \rangle$. 
    If assignment rather than accumulation is
    desired, {\sf GrB\_NULL} should be specified.

    \item[{\sf op}]   ({\sf IN}) Semiring used in the vector-matrix
    multiply.
    %: ${\sf op}=\langle \bDout({\sf op}),\bDin1({\sf op}),\bDin2({\sf op}),\oplus,\otimes,0 \rangle$.

    \item[{\sf u}]    ({\sf IN}) The GraphBLAS vector holding the values for
    the left-hand vector in the multiplication.

    \item[{\sf A}]    ({\sf IN}) The GraphBLAS matrix holding the values
    for the right-hand matrix in the multiplication.

    \item[{\sf desc}] ({\sf IN}) An optional operation descriptor. If
    a \emph{default} descriptor is desired, {\sf GrB\_NULL} should be
    specified. Non-default field/value pairs are listed as follows:  \\

    \hspace*{-2em}\begin{tabular}{lllp{2.7in}}
        Param & Field  & Value & Description \\
        \hline
        {\sf w}    & {\sf GrB\_OUTP} & {\sf GrB\_REPLACE} & Output vector {\sf w}
        is cleared (all elements removed) before the result is stored in it.\\

        {\sf mask} & {\sf GrB\_MASK} & {\sf GrB\_STRUCTURE}   & The write mask is
        constructed from the structure (pattern of stored values) of the input
        {\sf mask} vector. The stored values are not examined.\\

        {\sf mask} & {\sf GrB\_MASK} & {\sf GrB\_COMP}   & Use the
        complement of {\sf mask}. \\

        {\sf A}    & {\sf GrB\_INP1} & {\sf GrB\_TRAN}   & Use transpose of {\sf A}
        for the operation. \\
    \end{tabular}
\end{itemize}

\paragraph{Return Values}

\begin{itemize}[leftmargin=2.1in]
    \item[{\sf GrB\_SUCCESS}]         In blocking mode, the operation completed
    successfully. In non-blocking mode, this indicates that the compatibility 
    tests on dimensions and domains for the input arguments passed successfully. 
    Either way, output vector {\sf w} is ready to be used in the next method of 
    the sequence.

    \item[{\sf GrB\_PANIC}]           Unknown internal error.

    \item[{\sf GrB\_INVALID\_OBJECT}] This is returned in any execution mode 
    whenever one of the opaque GraphBLAS objects (input or output) is in an invalid 
    state caused by a previous execution error.  Call {\sf GrB\_error()} to access 
    any error messages generated by the implementation.

    \item[{\sf GrB\_OUT\_OF\_MEMORY}] Not enough memory available for the operation.

    \item[{\sf GrB\_UNINITIALIZED\_OBJECT}] One or more of the GraphBLAS objects 
    has not been initialized by a call to {\sf new} (or {\sf dup} for matrix or
    vector parameters).

    \item[{\sf GrB\_DIMENSION\_MISMATCH}] Mask, vector, and/or matrix 
    dimensions are incompatible.

    \item[{\sf GrB\_DOMAIN\_MISMATCH}]    The domains of the various vectors/matrices are
    incompatible with the corresponding domains of the semiring or
    accumulation operator, or the mask's domain is not compatible with {\sf bool}
    (in the case where {\sf desc[GrB\_MASK].GrB\_STRUCTURE} is not set).
\end{itemize}

\paragraph{Description}

{\sf GrB\_vxm} computes the vector-matrix product ${\sf w}^T = {\sf
u}^T \oplus . \otimes {\sf A}$, or, if an optional binary accumulation
operator ($\odot$) is provided, ${\sf w}^T = {\sf w}^T \odot
\left({\sf u}^T \oplus . \otimes {\sf A}\right)$ (where matrix {\sf A}
 can be optionally transposed).  Logically, this operation
occurs in three steps:
\begin{enumerate}[leftmargin=0.85in]
\item[\bf Setup] The internal vectors, matrices and mask used in the computation are formed and their domains/dimensions are tested for compatibility.
\item[\bf Compute] The indicated computations are carried out.
\item[\bf Output] The result is written into the output vector, possibly under control of a mask.
\end{enumerate}

Up to four argument vectors or matrices are used in the {\sf GrB\_vxm} operation:
\begin{enumerate}
	\item ${\sf w} = \langle \bold{D}({\sf w}),\bold{size}({\sf w}),\bold{L}({\sf w}) = \{(i,w_i) \} \rangle$
	\item ${\sf mask} = \langle \bold{D}({\sf mask}),\bold{size}({\sf mask}),\bold{L}({\sf mask}) = \{(i,m_i) \} \rangle$ (optional)
	\item ${\sf u} = \langle \bold{D}({\sf u}),\bold{size}({\sf u}),\bold{L}({\sf u}) = \{(i,u_i) \} \rangle$
	\item ${\sf A} = \langle \bold{D}({\sf A}),\bold{nrows}({\sf A}), \bold{ncols}({\sf A}),\bold{L}({\sf A}) = \{(i,j,A_{ij}) \} \rangle$
\end{enumerate}

The argument matrices, vectors, the semiring, and the accumulation operator (if provided) 
are tested for domain compatibility as follows:
\begin{enumerate}
	\item If {\sf mask} is not {\sf GrB\_NULL}, and ${\sf desc[GrB\_MASK].GrB\_STRUCTURE}$
    is not set, then $\bold{D}({\sf mask})$ must be from one of the pre-defined types of 
    Table~\ref{Tab:PredefinedTypes}.

	\item $\bold{D}({\sf u})$ must be compatible with $\bDin1({\sf op})$ of the semiring.

	\item $\bold{D}({\sf A})$ must be compatible with $\bDin2({\sf op})$ of the semiring.

	\item $\bold{D}({\sf w})$ must be compatible with $\bDout({\sf op})$ of the semiring.

	\item If {\sf accum} is not {\sf GrB\_NULL}, then $\bold{D}({\sf w})$ must be compatible with $\bDin1({\sf accum})$ and $\bDout({\sf accum})$ of the 
	accumulation operator and $\bDout({\sf op})$ of the semiring must be compatible with $\bDin2({\sf accum})$ of the accumulation operator.
\end{enumerate}
Two domains are compatible with each other if values from one domain can be cast 
to values in the other domain as per the rules of the C language.
In particular, domains from Table~\ref{Tab:PredefinedTypes} are all compatible 
with each other. A domain from a user-defined type is only compatible with itself.
If any compatibility rule above is violated, execution of {\sf GrB\_vxm} ends and 
the domain mismatch error listed above is returned.

From the argument vectors and matrices, the internal matrices and mask used in 
the computation are formed ($\leftarrow$ denotes copy):
\begin{enumerate}
	\item Vector $\vector{\widetilde{w}} \leftarrow {\sf w}$.

	\item One-dimensional mask, $\vector{\widetilde{m}}$, is computed from 
    argument {\sf mask} as follows:
	\begin{enumerate}
		\item If ${\sf mask} = {\sf GrB\_NULL}$, then $\vector{\widetilde{m}} = 
        \langle \bold{size}({\sf w}), \{i, \ \forall \ i : 0 \leq i < 
        \bold{size}({\sf w}) \} \rangle$.

		\item If {\sf mask} $\ne$ {\sf GrB\_NULL},  
        \begin{enumerate}
            \item If ${\sf desc[GrB\_MASK].GrB\_STRUCTURE}$ is set, then
            $\vector{\widetilde{m}} = 
            \langle \bold{size}({\sf mask}), \{i : i \in \bold{ind}({\sf mask}) \} \rangle$,
            \item Otherwise, $\vector{\widetilde{m}} = 
            \langle \bold{size}({\sf mask}), \{i : i \in \bold{ind}({\sf mask}) \wedge
            ({\sf bool}){\sf mask}(i) = \true \} \rangle$.
        \end{enumerate}

		\item	If ${\sf desc[GrB\_MASK].GrB\_COMP}$ is set, then 
        $\vector{\widetilde{m}} \leftarrow \neg \vector{\widetilde{m}}$.
	\end{enumerate}

	\item Vector $\vector{\widetilde{u}} \leftarrow {\sf u}$.

	\item Matrix $\matrix{\widetilde{A}} \leftarrow {\sf desc[GrB\_INP1].GrB\_TRAN} \ ? \ {\sf A}^T : {\sf A}$.
\end{enumerate}

The internal matrices and masks are checked for shape compatibility. The following 
conditions must hold:
\begin{enumerate}
	\item $\bold{size}(\vector{\widetilde{w}}) = \bold{size}(\vector{\widetilde{m}})$.

	\item $\bold{size}(\vector{\widetilde{w}}) = \bold{ncols}(\matrix{\widetilde{A}})$.

	\item $\bold{size}(\vector{\widetilde{u}}) = \bold{nrows}(\matrix{\widetilde{A}})$.
\end{enumerate}
If any compatibility rule above is violated, execution of {\sf GrB\_vxm} ends and 
the dimension mismatch error listed above is returned.

From this point forward, in {\sf GrB\_NONBLOCKING} mode, the method can 
optionally exit with {\sf GrB\_SUCCESS} return code and defer any computation 
and/or execution error codes.

We are now ready to carry out the vector-matrix multiplication and any additional 
associated operations.  We describe this in terms of two intermediate vectors:
\begin{itemize}
    \item $\vector{\widetilde{t}}$: The vector holding the product of vector
    $\vector{\widetilde{u}}^T$ and matrix $\matrix{\widetilde{A}}$.
    \item $\vector{\widetilde{z}}$: The vector holding the result after 
    application of the (optional) accumulation operator.
\end{itemize}

The intermediate vector $\vector{\widetilde{t}} = \langle
\bDout({\sf op}), \bold{ncols}(\matrix{\widetilde{A}}),
\{(j,t_j) : \bold{ind}(\vector{\widetilde{u}}) \cap
\bold{ind}(\matrix{\widetilde{A}}(:,j)) \neq \emptyset \} \rangle$
is created.  The value of each of its elements is computed by 
\[t_j = \bigoplus_{k \in \bold{ind}(\vector{\widetilde{u}}) \cap
\bold{ind}(\matrix{\widetilde{A}}(:,j))} (\vector{\widetilde{u}}(k)
\otimes \matrix{\widetilde{A}}(k,j)),\] where $\oplus$ and $\otimes$
are the additive and multiplicative operators of semiring {\sf op},
respectively.



\appendix
\chapter{Revision history}
\label{Chp:RevHistory}
%--------------------------------------------------------------

This document defines the LAGraph 1.0 release and hence one could argue that there should not be a
revision history just yet.    Early pre-release versions of LAGraph, however, have been heavily used.  
We therefore need to summarize the key changes from the pre-release version of LAGraph and 
the official, 1.0 release.  
Changes in 1.0 (Released: 12 September 2022):
\begin{itemize}
\item We did a global redefinition of return codes to be more consistent and to mesh better with the GraphBLAS
return codes.
\item In the pre-release LAGraph library, we included type information on the LAGraph graph object.  We have
deprecated this feature since it is safer to use the type introspection from GraphBLAS than to carry distinct type
information inside the LAGraph  object.
\end{itemize}


%--------------------------------------------------------------

\chapter{Examples}
\label{Chp:Examples}

Text to introduce the examples.

\pagebreak
\nolinenumbers
\section{Example: Compute the page rank of a graph using LAGraph.}
{\scriptsize
\lstinputlisting[language=C,numbers=left]{ExPageRank.c}
}
\vfill

\pagebreak
\nolinenumbers
\section{Example: Apply betweenness centrality algorithm to a Graph using LAGraph}
{\scriptsize
\lstinputlisting[language=C,numbers=left]{ExBetween.c}
}
\vfill

\linenumbers


%\def\IEEEbibitemsep{3pt plus .5pt}
%\bibliographystyle{IEEEtran}
%\bibliography{refs}

\end{document}
